

Up to now we have always considered the problem of recovering an unknown function from a fixed set of information about it, namely the values that the function assumes on a given set of points.  But never we considered what happens to the interpolant when our knowledge on the sampled function gradually increases, that is when we have access to more and more sampled data.  It is reasonable to expect that the reconstructed function becomes increasingly more similar to the original function. This is the topic of concern of this section.


Let $X$ and $\.Y$ be two sets of data sites such that~$X\subset Y$.  Given a function~$f$ in the native space~$\Cal N_Φ$ of a symmetric positive definite kernel~$Φ$, we can compute both of the interpolants $s_{X,f}\in\Cal S_{Φ, X}$ and~$s_{Y,f}\in\Cal S_{Φ, Y}$.  It is easy to see that, as expected,  $s_{Y,f}$ is no worse than~$s_{X,f}$ at approximating~$f$, meaning that 
$$
\norm{\.f - s_{Y,f}}_Φ \leq \norm{\.f - s_{X,f}}_Φ.
$$
This is an immediate consequence of the inclusion~$\Cal S_{Φ, X}\subset\Cal S_{Φ, Y}$ and the property~\ref[mindist] of minimum distance satisfied by~$s_{Y,f}$.  If we then combine the Pythagorean identities~\ref[pythag] satisfied by $s_{X,f}$ and~$s_{Y,f}$ with the above inequality we further obtain that
$$
\norm{s_{X, f}}_Φ\leq\norm{s_{Y,f}}_Φ\leq\norm{f}_Φ.
$$
%So, when additional information on the sampled function~$f$ is available, not only the interpolant  provides a better approximation, but also its native space norm increases. 

There is actually an explicit expression for the increase in norm of the interpolant, as shown by Shaback and~Werner~\cite[shaback_2006].  Specifically, in the simple case where $Y=X\cup\{x\}$, that is when just one point is added to the data sites, the following result holds.

\preskip
\theorem
Let~$f$ be a function in the native space~$\Cal N_Φ$ of a symmetric positive definite kernel~$Φ:\Omega\times\Omega\to\R$.  If $X\subset\Omega$ is a set of locations and~$x\in\Omega$ is a point {\em not}  belonging to~$X$, then the norm of the interpolant~$s_{X\cup\{x\},f}\in\Cal S_{Φ, X\cup\{x\}}$ of~$f$ on the extended set of locations~$X\cup\{x\}$ is related to the norm of the interpolant~$s_{X, f}\in\Cal S_{Φ, X}$ of~$f$ on~$X$ by the expression
$$
\norm{s_{X\cup\{x\}, f}}_Φ^2 = \norm{s_{X, f}}_Φ^2 + {(f(x) - s_{X, f}(x))^2\over P^2_X(x)},
$$
where $P_X$ is the power function associated to~$X$.

\proof
Let $X=\{\range x_N/\}$, and define the  matrix~$\Bm A$ and the vectors~$\Bm u(x), \Bm b(x)$ as in section~\ref[errorsec].
The interpolants $s_{X, f}$ and~$s_{X\cup\{x\},f}$ can be expressed as linear combinations of the standard basis functions of $\Cal S_{Φ, X}$ and~$\Cal S_{Φ, X\cup\{x\}}$ respectively. Explicitly,
$$
s_{X,f} = \sum_{j = 1}^N\alpha_j Φ(\cdot, x_j),  \qquad
s_{X\cup\{x\}, f} = \sum_{j=1}^N\beta_j Φ(\cdot, x_j) + \gamma\, Φ(\cdot, x),
$$
for some coefficients~$\{\alpha_j\}_{j\in\{1,2,\dots,N\}}$, $\{\.\beta_j\}_{j\in\{1,2,\dots,N\}}$ and~$\gamma$.
If we define the vectors 
$$
\Bm \alpha \coloneq (\range\alpha_N/)^T \quad\hbox{and}\quad \Bm \beta \coloneq (\range\beta_N/)^T,
$$
 then the evaluations of the two~interpolants at the points $\{x_j\}_{j\in\{1,2,\dots,N\}}$ of~$X$ have the expressions
$$
s_{X,f}(x_j) =\Bm b(x_j)^T\!\Bm\alpha\quad\hbox{and}\quad s_{X\cup\{x\},f}(x_j)  = \Bm b(x_j)^T\!\Bm\beta + \gamma\,Φ(x_j,x),
$$
which can be collected in a compact way as
$$
\pmatrix{
s_{X,f}(x_1) \cr
s_{X,f}(x_2) \cr
\vdots 	    \cr
s_{X,f}(x_N)
} = \Bm A \Bm \alpha\., \quad\hbox{and}\quad
\pmatrix{
s_{X\cup\{x\},f}(x_1) \cr
s_{X\cup\{x\},f}(x_2) \cr
\vdots 	    \cr
s_{X\cup\{x\},f}(x_N)
} = \Bm A \Bm \beta + \gamma\Bm b(x),
$$
simply by noticing that~$\Bm b(x_j)^T$ is the $j$-th row of the matrix~$\Bm A$.  Since by definition the interpolants $s_{X, f}$ and~$s_{X\cup\{x\}, f}$ assume the same value at each point of~$X$,  the relation
$$
\Bm A \Bm \alpha = \Bm A \Bm \beta + \gamma \Bm b(x)
$$
must hold.  After having multiplied both sides by the inverse of~$\Bm A$ and having rearranged its terms, this equation gives an expression for the vector of coefficients~$\Bm \beta$, that is
$$
\eqalign{
\Bm \beta &= \Bm \alpha - \gamma \Bm A^{-1} \Bm b(x) \cr
		  &= \Bm \alpha - \gamma\. \Bm u(x). 
} \eqno\hbox{($\lower.6ex\hbox{*}$)}
$$

We now compute the squares of the native space norms of $s_{X, f}$ and~$s_{X\cup\{x\}, f}$ by expanding the scalar product as it is defined in~\ref[form].  For~$s_{X, f}$ we get
$$
\norm{s_{X,f}}_Φ^2 = \bnorm{\sum_{j=1}^N \alpha_j Φ(\cdot, x_j)}_Φ^2 = \sum_{j=1}^N\sum_{k=1}^N \alpha_j \alpha_k Φ(x_j, x_k) =  \Bm \alpha^T\! \Bm A \.\Bm \alpha,
$$
while for~$s_{X\cup\{x\}, f}$ analogous computations yield
$$
\eqalign{
\norm{s_{X\cup\{x\}, f}}_Φ^2 &= \bnorm{\sum_{j = 1}^N \beta_j Φ(\cdot, x_j)}_Φ^2 + 2\,\bform{\sum_{j=1}^N \beta_j Φ(\cdot, x_j)}{\gamma\, Φ(\cdot, x)}_Φ + \norm{\gamma\, Φ(\cdot, x)}_Φ^2 \cr
&= \sum_{j=1}^N\sum_{k=1}^N \beta_j \beta_k Φ(x_j, x_k) + 2\gamma\sum_{j = 1}^N \beta_j Φ(x_j, x) + \gamma^2 Φ(x,x) \cr
&= \Bm \beta^T\! \Bm A \.\Bm \beta + 2 \gamma\. \Bm b(x)^T \Bm \beta + \gamma^2 Φ(x,x).
}
$$
Then, the substitution~(\lower.6ex\hbox{*}) for~$\Bm \beta$ allows us to express the square of the norm of~$s_{X\cup\{x\}, f}$ as
$$
\eqalign{
\norm{s_{X\cup\{x\}, f}}_Φ^2 &= (\Bm \alpha - \gamma\Bm u(x))^T\Bm A\, (\Bm \alpha - \gamma \Bm u(x)) + 2\gamma\. \Bm b(x)^T\! (\Bm \alpha - \gamma \Bm u(x)) + \gamma^2 Φ(x,x) \cr
&=  \Bm \alpha^T\!\Bm A\.\Bm \alpha +\gamma^2 \bigl[Φ(x,x) - 2\Bm b(x)^T\Bm u(x) + \Bm u(x)^T\!\Bm A\. \Bm u(x) \bigr] \cr
&= \norm{s_{X,f}}_Φ^2 + \gamma^2 P^2_X(x),
}\eqno\hbox{($\lower.6ex\hbox{*}\lower.6ex\hbox{*}$)}
$$
where in the last step we recognised the power function from expression~\ref[pow1].

It is finally possible to find an expression for~$\gamma$ by computing the difference of the interpolants at the adjoined point~$x$.  In fact,
$$
\eqalign{
s_{X\cup\{x\}, f}(x) - s_{X, f}(x) &= \Bm b(x)^T\Bm \beta + \gamma\, Φ(x,x) - \Bm b(x)^T\Bm\alpha \cr
&= \Bm b(x)^T(\Bm \alpha - \gamma\Bm u(x)) + \gamma\,Φ(x,x) - \Bm b(x)^T\Bm\alpha \cr
&= \gamma\bigl[Φ(x,x) - \Bm b(x)^T\Bm u(x)\bigr] \cr
&= \gamma P^2_X(x),
}
$$
where this time we used expression~\ref[pow2] to recognise~$P_X(x)$, and hence
$$
\gamma = {s_{X\cup\{x\}, f}(x) - s_{X, f}(x)\over P^2_X(x)}.
$$
The substitution of this value of~$\gamma$ into expression~(\lower.6ex\hbox{*}\lower.6ex\hbox{*}) concludes the proof.~\QED
\postskip


This theorem says that if the interpolant~$s_{X,f}$ already makes a correct prediction for a point~$x\notin X$, in the sense that $s_{X,f}(x)=f(x)$, then the addition of the point~$x$ has no effect on the norm of the interpolant, in fact
$$
\norm{s_{X\cup\{x\},f}}_Φ^2=\norm{s_{X,f}}_Φ^2+ {(f(x) - s_{X, f}(x))^2\over P^2_X(x)}=\norm{s_{X,f}}_Φ^2.
$$
But this is something that we already knew, since in this case the interpolant itself doesn't change, because $s_{X,f}$, just like $s_{X\cup\{x\},f}$, is a function in~$\Cal S_{Φ, X\cup\{x\}}$ which interpolates~$f$ at the data sites~$X\cup\{x\}$, and the solution of the interpolation problem is unique in~$\Cal S_{Φ, X\cup\{x\}}$.

On the other hand, if $s_{X,f}$ predicts at the point~$x$ a value~$s_{X,f}(x)$ different from $f(x)$, then the difference of the squares of the norms, which by the above theorem can be computed as
$$
\norm{s_{X\cup\{x\},f}}_Φ^2 -\norm{s_{X,f}}_Φ^2= {(f(x) - s_{X, f}(x))^2\over P^2_X(x)},
$$
can be big not only if the difference~$f(x)-s_{X,f}(x)$ is big in absolute value, but also if the value~$P_X(x)$ is small, which happens when the new point~$x$ is close to one of the points of~$X$.% whereas the difference of the squares of the norms has value approximately equal to~$((x)-s_{X,f}(x))^2$ when $x$ is sufficiently far from the points of~$X$.
In fact, in section~\ref[errorsec] we saw that the power function~$P_X$ associated to a set~$X$ of data sites is continuous (if the kernel is continuous) and has value zero at each point of~$X$,  



Therefore, computing the difference of the squares of the norms is a geometrically correct way of saying how much s predicts


% TODO: the norm of the interpolating converges, but not necessarily to the norm of the function.  Example of this.   The location/distribution of the data sites is important.  Results for pointwise convergence.







\comment

Let us now consider a nested infinite sequence~$\{ Y_n\}_{n\in\N}$ of sets of locations,
$$
X=Y_1\subset Y_2 \subset Y_3 \subset \cdots \subset\R^d
$$
and the associated infinite sequence~$\{s_{Y_n, f}\}_{n\in\N}$ of interpolants of a function~$f\in\Cal N_Φ$.

\endcomment

