

We have seen some examples of interpolating spaces generated by so called {\em radial basis functions}, i.e., spaces
$$
\Cal S_{Φ,\. X}\coloneq\langle\range B_N/\rangle
$$
whose basis elements~$\range B_N/$ have common expressions
$$
B_k(x) = Φ(x,x_k)=\phi(\norm{x-x_k})\mathbox{,\quad $k\in\{1,2,\dots,N\}$}
$$
for a single univariate function~$\phi:[0,\infty)\to\R$.
Furthermore, we showed that if we choose a function~$Φ$ which is positive definite, then the so generated space~$\Cal S_{Φ,\. X}$ is nondegenerate for~$X=\{\range x_N/\}\subset\R^d$, meaning that the interpolation of values in this space at the locations~$X$ leads to a nonsingular  interpolation matrix~$\Bm A_{Φ,\. X}$.  The interpolating spaces that we managed to find depend on the set~$X$ of data sites, as prescribed by the Mairhuber-Curtis theorem (\ref[mcth]), nevertheless they have the nice property of being all generated by one single {\em kernel function~$Φ$}.
Now it’s time to collect all these spaces~$\Cal S_{Φ,\. X}$ for a fixed~$Φ$ and varying~$X$ into a bigger space in a very useful way.  We will see that the positive definiteness of~$Φ$ is crucial for the construction of this new space.

Let $\Omega$ be a domain in~$\R^d$ and $Φ:\,\Omega\times\Omega\to\R$ be a positive definite function.  Define~$F_Φ$ as the smallest linear space of functions~$\Omega\times\Omega\to\R$ containing the union of the spaces~$\Cal S_{Φ,\. X}$ taken over all~$X$s, i.e.,
$$
F_{Φ}=\langle\,Φ(\cdot,y)\mathbox{,\quad $y\in\Omega$}\,\rangle\mathbox,
$$
which is the set of all finite combinations of the functions~$\{Φ(\cdot,y)\}_{y\in\Omega}$.
Notice that this space is {\em infinite dimensional} if $Φ$ is positive definite, because we already know that in this case each of its subspaces~$\Cal S_{Φ,\. X}$ has dimension equal to the number of (distinct) points in~$X$.

The space~$F_Φ$ may be endowed with the form~$\form\;\;_Φ$ defined by
$$
\Bform{\sum_{j=1}^Nα_jΦ(\cdot, x_j)}{\sum_{k=1}^Mβ_kΦ(\cdot,y_k)}_Φ\coloneq \sum_{j=1}^N\sum_{k=1}^Mα_j β_kΦ(x_j,y_k)\mathbox.\eqmark[form]
$$
This  form is well defined because the basis functions~$\{Φ(\cdot,y)\}_{y\in\Omega}$ are linearly independent, which implies that each function in~$F_Φ$ can be express  in a {\em unique} way as a finite linear combination of them.

\preskip
\proposition
If \,$Φ:\,\Omega\times\Omega\to\R$ is a symmetric positive definite kernel, then~$\form\;\;_Φ$ defines a {\em scalar product} on~$F_Φ$.
\proof
The form $\form\;\;_Φ$ is obviously {\em bilinear} from its definition, and it is {\em symmetric} if $Φ$ is symmetric.   Moreover, if we take an arbitrary~$f=\sum_{j=1}^Nα_j Φ(\cdot,x_j)$ in~$F_Φ$, except the zero function, then  the positive definiteness of~$Φ$ implies that
$$
\form ff_Φ=\sum_{j=1}^N\sum_{k=1}^Nα_j α_k Φ(x_j,x_k) > 0\mathbox.\quad\QED
$$
\postskip

\noindent If the kernel is positive definite and radial, i.e., it is of the kind~\ref[phi], then it is automatically symmetric, and~$\form\;\;_Φ$ is a scalar product.  If we have a scalar product on~$F_Φ$, we can also define a {\em norm} in the usual way,
$$
\norm{f}_Φ=\sqrt{\form ff_Φ}\mathbox{,\quad $f\in F_Φ$,}
$$
making~$F_Φ$ a {\em normed space}.  Even if we don’t specify it, we will always assume that the kernel~$Φ$ is symmetric and positive definite, so that the bilinear form is a scalar product and the norm is defined.

Let us now call~$\{\delta_y\}_{y\in\Omega}$ the {\em point-evaluation} functionals.  For each~$y\in\Omega$ the functional~$\delta_y$ is defined as the function~$F_Φ\to\R$ which maps each~$f\in F_Φ$ to its evaluation at the point~$y$,
$$
\delta_y(f) = f(y)\mathbox{,\quad for each~$f\in F_Φ$.}
$$
One important property of these functionals is the fact that they can be represented by an actual function in~$F_Φ$ through the inner product~$\form\;\;_Φ$.  More precisely,

\preskip
\theorem[(Reproducing property)]
For each~$y\in\Omega$ the point-evaluation functional~$\delta_y: F_Φ\to\R$ is represented by the function~$Φ(\cdot,y)\in F_Φ$ by means of the inner product~$\form\;\;_Φ$, i.e.,
$$
\delta_y(f) = \form {Φ(\cdot,y)}f_Φ\mathbox{,\quad for each~$f\in F_Φ$.}\eqmark[repr]
$$
Moreover, these functionals belong the  {\em topological dual\nobreak\,}\fnote{The topological dual of~$F_Φ$ is the set of all the linear functionals~$F_Φ\to\R$ that  continuous with respect to the topology induced by the norm~$\norm\;_Φ$.} $F_Φ^*$ of~$F_Φ$ and they are {\em linearly independent}. 


\proof
The {\em reproducing property}~\ref[repr] of the kernel~$Φ$ holds simply because of how the scalar product is defined.  In fact, if $f=\sum_{j=1}^Nα_j Φ(\cdot,x_j)$ then
$$
\form {Φ(\cdot,y)}f_Φ = \sum_{j=1}^Nα_j Φ(y,x_j)=f(y)=\delta_y(f)\mathbox.
$$
The continuity of~$\,\delta_y$ with respect to the norm~$\norm\;_Φ$ follows immediately from the continuity of the scalar product~$\form\;\;_Φ$.  Finally, the functionals~$\{\delta_y\}_{y\in\Omega}$ are linearly independent in~$F_Φ^*$ because the functions~$\{Φ(\cdot,y)\}_{y\in\Omega}$ are linearly independent in~$F_Φ$.~\QED
\postskip



The inner product space~$F_Φ$ need not be complete, which means that there can be Cauchy sequences in~$F_Φ$ that do not converge.  It is possible anyhow to build a {\em completion}\fnote{Details about the construction of the completion of a Hilbert space are present, for instance, in Stein~\cite[stein_2009].} for it, i.e., it is possible to construct an inner product  space~$\Cal F_Φ$ and a {\em linear injective} map~$A :F_Φ\rightarrowtail\Cal F_Φ$ such that 
\begitems
* $\form{A(x)}{A(y)}_{\Cal F_Φ} = \form xy_{F_Φ}\mathbox{,\quad for all~$x,y\in F_Φ$;}$
* $A(F_Φ)$ is {\em dense} in~$\Cal F_{Φ}$;
* $\Cal F_{Φ}$ is a {\em Hilbert space}, i.e., it is complete.
\enditems
  Here we used the notations $\form\;\;_{F_Φ}$ and~$\form\;\;_{\Cal F_Φ}$ to distinguish between the scalar products on $F_Φ$ and~$\Cal F_Φ$ respectively, but we can denote both of them simply by~$\form\;\;_{Φ}$, as it is clear which one we refer to just by looking at the paired functions.

Unfortunately, the completion~$\Cal F_Φ$ of the space~$F_Φ$ is an abstract space, because its elements  are equivalence classes of Cauchy sequences of functions in~$F_Φ$. We must therefore reinterpret those elements as standard functions.    This is actually possible by considering in a certain sense the extensions of the point-evaluation functionals to all of~$\Cal F_Φ$.  Precisely, let us denote by~$\R^\Omega$ the set of all functions from $\Omega$ to~$\R$, and then define the map~$R:\Cal F_Φ\to\R^\Omega$ as
$$
(Rb)(x)\coloneq \form{A(Φ(\cdot,x))}b_Φ\mathbox{,\quad for all~$x\in\Omega$,}
$$
which maps an abstract element~$b\in\Cal F_Φ$ to an actual function~$Rb\in\R^\Omega$ that can be evaluated at each point~$x\in\Omega$.
\nobreak
\preskip
\lemma The above defined map~$R$ is injective.
\proof
If $b\in\Cal F_Φ$ is such that~$Rb=0$, then 
$$
 \form{A(Φ(\cdot,x))}b_Φ=0\mathbox{,\quad for all~$x\in\Omega$,}
$$
or, in other words, $b$ is perpendicular to all of~$A(F_Φ)$, since the space~$F_Φ$ is generated by the elements~$\{Φ(\cdot,x)\}_{x\in\Omega}$.  For the fact that $A(F_Φ)$ is dense in the Hilbert space~$\Cal F_Φ$, this implies that~$b=0$.~\QED
\postskip

Now, for every function $f\in F_Φ$ we have that $R(A(f))=f$.  In fact, for each~$x\in\Omega$,
$$
\eqalign{
R(A(f))(x)&=\form{A(Φ(\cdot,x))}{A(f)}_{Φ}  \cr
 	        &=\form{Φ(\cdot,x)} f_Φ                 \cr
 	        &= f(x)\mathbox.}
$$
This means that, for each~$f\in F_Φ$, the corresponding abstract element~$A(f)\in\Cal F_Φ$ gets realised again as the same function~$f$ by the map~$R$.  This map~$R$, being defined on all of~$\Cal F_Φ$, allows us to interpret as standard functions also the other elements of~$\Cal F_Φ$ that do not come through~$A$  from functions of~$F_Φ$. 

\preskip
\definition We define the {\em native Hilbert space}~$\Cal N_Φ$ deriving from a symmetric positive definite kernel~$Φ:\,\Omega\times\Omega\to\R$ as the space
$$
\Cal N_Φ\coloneq R(\Cal F_Φ)
$$
with the inner product
$$
\form fg_{\Cal N_Φ}\coloneq\form{R^{-1}(f)}{R^{-1}(g)}_Φ\mathbox{,\quad $f,g\in\Cal N_Φ$.}
$$
\postskip

\noindent The native Hilbert space of~$Φ$ is thus the concrete version of the completion~$\Cal F_Φ$ of the space of functions~$F_Φ$.  In fact, $\Cal N_Φ$ is a Hilbert space such that
$$
F_Φ\subseteq \Cal N_Φ \subseteq \R^\Omega
$$
and  $F_Φ$ is dense in~$\Cal N_Φ$.  Moreover, the inner product~$\form\;\;_{\Cal N_Φ}$ and hence the corresponding norm~$\norm{\;}_{\Cal N_Φ}$ by definition extend respectively the inner product and the norm of~$F_Φ$---we can again denote them  as $\form\;\;_Φ$ and~$\norm\;_Φ$  for simplicity.  Finally, the point evaluation functionals, previously defined only on~$F_Φ$, can be naturally extended to all of~$\Cal N_Φ$, preserving their continuity.  In fact, if $f\in\Cal N_Φ$, then 
$$
\eqalign{ \delta_x(f)&\coloneq\form{Φ(\cdot, x)} f_Φ \cr
                                       &= \form{R(A(Φ(\cdot, x))}{R(b)}_Φ\rlap{, \quad for some~$b\in\Cal F_Φ$} \cr
                                       &= \form{A(Φ(\cdot,x))}b_Φ \cr
                                       &=(Rb)(x) = f(x)\mathbox.
}
$$


 Not all functions belong to the native space---the class of functions belonging to~$\Cal N_Φ$ depends on the properties of the {\em reproducing\fnote{A function~$Φ:\,\Omega\times\Omega\to\R$ is called {\em reproducing kernel} for the Hilbert space~$\Cal H$ if it satisfies the reproducing property~\ref[repr].  It is easy to show that the reproducing kernel of a Hilbert space is unique.  It can also be shown, by using the Riesz representation theorem (Stein~\cite[stein_2009]), that the existence of a reproducing kernel is equivalent to the continuity of the point-evaluation functionals, which are linearly independent if and only if the kernel is positive definite (Wendland~\cite[wendland_2004], chapter~10).} kernel~$Φ$}.  For instance the continuity of the kernel is carried over to the native space.

\preskip
\theorem
A symmetric positive definite kernel~$Φ$ which is also {\em continuous} has a native space  made of continuous functions,
$$
\Cal N_Φ\subseteq\Cal C(\Omega)\mathbox.
$$
\proof
If $x,y$ are points in~$\Omega$ and $f$~is a function in the native space, then
$$
\eqalign{
|\.f(x)-f(y)|&=|\form{Φ(\cdot,x)}{f}_Φ - \form{Φ(\cdot, y)}{f}_Φ| \cr
	        &=|\form{Φ(\cdot,x)-Φ(\cdot,y)}{f}_Φ|\cr
                 &\leq\norm{Φ(\cdot,x)-Φ(\cdot,y)}_Φ\,\norm{\.f}_Φ \mathbox,}
$$
by having applied the Cauchy-Schwarz inequality.  The right term of the inequality tends to zero as~$y\to x$, since the continuity of $Φ$ implies that
$$
\lim_{y\to x}\norm{Φ(\cdot,x)-Φ(\cdot,y)}_Φ^2=\lim_{y\to x}\bigl(Φ(x,x)-2\.Φ(x,y)+Φ(y,y)\bigr)=0\mathbox.\quad\QED
$$
\postskip 

\noindent An analogous result holds also for differentiability---a continuously differentiable kernel has a native space made of continuously differentiable functions (Shaback~\cite[shaback_1999]).

On the native space~$\Cal N_Φ$ of a symmetric positive definite kernel~$Φ$ there are two notions of {\em convergence}.  One is the convergence in the norm of the space, the other is the pointwise convergence, since the elements of~$\Cal N_Φ$ are actual functions that can be evaluated at each point of their domain.  These two notions are related.
\preskip
\theorem
If $\{\.f_n\}_{n\in\N}$ is a sequence of functions in~$\Cal N_Φ$ which converges in the native space norm to a function~$f\in \Cal N_Φ$, then
$$
\lim_{n\to\infty}\, f_n(x) = f(x)\mathbox{,\quad for each~$x\in\Omega$.}
$$
\proof
This is simply a consequence of the continuity of the point-evaluation functionals. Explicitly, if $f_n$ converges to $f$ in~$\Cal N_Φ$, then $\delta_x(f_n)$ converges to~$\delta_x(f)$ in~$\R$, because the functional~$\delta_x:\Cal N_Φ\to\R$ is continuous.~\QED
%\postskip
