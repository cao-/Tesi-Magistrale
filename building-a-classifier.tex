
The ideas developed in the previous section can be used to try and distinguish samples that come from a regular function from those that come from a discontinuous function.  The regularity classifier that we are going to define will be the main toll necessary to later develop an algorithm for the interpolation of a discontinuous function.


In what follows, even if not always strictly necessary, we will assume that  the kernel~$Φ$ is radial, i.e., of the form~\ref[phi] for some univariate positive definite function~$\phi$.  This applies also to the notations, with the letter~$\phi$ replacing the uppercase letter~$Φ$.  Let $X=\{\range x_N/\}$ be a set of data sites and $f$ a function that is no more assumed to belong to the native space of~$\phi$.  Suppose that $f$ has been sampled at~$X$ and the sampling is dense enough to catch the relevant features of~$f$ on its considered domain.  Our goal is to recognise whether or not $f$ is discontinuous just by looking at~$X$ and the corresponding sampled function values~$f(X)$.



Each index~$k\in\{1,2,\dots,N\}$ determines a splitting of the data sites into two parts, which are~$X\setminus\{x_k\}$ and~$\{x_k\}$. To simplify the notation, in this context we let
$$
X^{(k)\!}\coloneq X\setminus\{x_k\},\quad s\coloneq s_{X,f}\quad\mathbox{and}\quad s^{(k)\!}\coloneq s_{X^{(k)\!},f}.
$$
If the function~$f$ that generated the given sampling~$f(X)$ is a well behaved function, then we should expect that, for each~$k$, the interpolant~$s^{(k)}\!$ predicts quite accurately the value~$f(x_k)$ at the point~$x_k$; instead, if $f$ is a discontinuous function, then there should be some indices~$k$ such that the prediction made by~$s^{(k)}\!$ for the point~$x_k$ is wrong.  Following the considerations of the previous section, we shall consider, for each~$k$, the quantity
$$
U_k\coloneq\norm s_\phi^2-\norm{s^{(k)\!}}^2_\phi = {(f_k-s^{(k)\!}(x_k))^2\over P_{X^{(k)\!}}^2(x_k)},
$$
which is small when the data~$(x_k, f(x_k))$ follows the model~$s$, which has been trained\fnote{Using the language of machine learning, in this case $(X^{(k)\!}, f(X^{(k)\!\.}))$ is the {\em training set}, while $(x_k, f(x_k))$ is the {\em test set}.} with~$(X^{(k)}, f(X^{(k)}))$,  and it is large otherwise.

It is of our interested to compute the vector
$$
\Bm U\coloneq (\range U_N/),
$$
which collects together all the quantities~$U_k$ that have just been defined.  This vector is closely related to the leave-one-out {\em cross-validation} (LOOCV) vector
$$
\Bm E\coloneq (\range E_N/),
$$
whose elements are defined as~$E_k\coloneq f(x_k)-s^{(k)\!}(x_k)$, and which is typically used to find the appropriate shape parameter~$ε$ for the interpolation with  a positive definite radial function~$\phi_ε$ as defined in~\ref[phiepsilon]. The value of~$ε$ is chosen such that the $p$-norm
$$
\norm{\Bm E}_p\coloneq\cases{\bigl(\sum_{j=1}^N |x_j|^p\bigr)^{1/p}, &\quad if $p\geq1$\cr
  \max_{j\in\{1,2,\dots,N\}}|x_j|, &\quad if $p=\infty$} \eqmark[p-norm]
$$
of the $ε$-dependent vector~$\Bm E$ is minimised, with $p$ usually being either $1$, $2$, or~$\infty$.


The cross-validation vector~$\Bm E$ wouldn't be of much practical utility if there weren't an efficient way to compute it.  Even if apparently its computation requires to find the $N$ interpolants~$\{s^{(k)}\}_{k\in\{1,2,\dots, N\}}$, and hence to invert $N$~matrices of size~$(N-1)\times (N-1)$, Rippa~\cite[rippa_1999] recognised that there is actually an efficient way to compute~$\Bm E$ by inverting only one matrix of size~$N\times N$, namely the interpolation matrix~$\Bm A$ associated to the full set~$X$ of data sites.  By following an analogous reasoning, we can show that also the vector~$\Bm U$ can be computed in an efficient way. But first we need to report the computations made by Rippa. 

\preskip
\theorem[(Rippa)]
Let $X=\{\range x_N/\}\subset\Omega\subseteq\R^d$ be a set of data sites, $\phi:[0,\infty)\to\R$ a radial kernel which is positive definite on~$\R^d$, and
$$
\Bm f\coloneq (\range f_N/)^T = (f(x_1), f(x_2),\dots f(x_N))^T
$$
a vector of values sampled from any unknown function~$f\in\R^\Omega$.
Then the elements~$E_k=f_k-s^{(k)\!}(x_k)$ of the cross-validation vector~$\Bm E$ can be alternatively computed as
$$
E_k = {α_k\over C_{k,k}},
$$
where $\Bm C$ is the inverse of the interpolation matrix~$\Bm A=\Bm A_{\phi,X}$ associated to the kernel~$\phi$ and the set~$X$ of locations, and $\Bm α$ is the vector of coefficients of the unique function~$s\in\Cal S_{\phi,X}$ that interpolates $f$ on~$X$, that is the solution of the linear sistem~$\Bm{Aα} = \Bm f$. 
\proof
For each point~$x\in\Omega$ let $\Bm t(x)$ be the evaluation at~$x$ of the vector of kernel translates centred at the points of~$X$, that we already defined in section~\ref[errorsec], namely
$$
\Bm t(x)\coloneq (\phi(\norm{x-x_1}), \phi(\norm{x-x_2}), \dots, \phi(\norm{x-x_N}))^T.
$$
Define then, for each~$k\in\{1,2,\dots, N\}$, the vectors $\Bm t^{(k)\!}(x)$ and~$\Bm f\,^{(k)\!\!}$ of length~$N-1$ by removing the $k$-th element from the vectors $\Bm t(x)$ and~$\Bm f$ respectively.  Similarly, define the matrix~$\Bm A^{(k)\!\!}$ of size~$(N-1)\times(N-1)$ by removing from~$\Bm A$ both its $k$-th~row and $k$-th~column.  If $\Bm β^{\lbrace k\rbrace\!}$ is the solution of the linear system~$\Bm A^{(k)}\Bm β^{\lbrace k\rbrace\!} = \Bm f\,^{(k)\!}$, then the function~$s^{(k)\!}\in \Cal S_{\phi,X^{(k)\!}}$ that interpolates~$f$ on the restricted set~$X^{(k)\!}$ of data sites can be expressed as
$$
s^{(k)\!}(x) = \Bm t^{(k)\!}(x)^T \Bm β^{\lbrace k\rbrace},\quad x\in\Omega.
$$

If $\Bm C^{\.[k]}$ and~$\Bm I^{\.[k]}$ denote the $k$-th columns of the matrices $\Bm C$ and~$\Bm I$ respectively, then for each~$k$ the relation
$$
\Bm A\Bm C^{\.[k]}=\Bm I^{\.[k]} \eqmark[ACI]
$$
must hold, since $\Bm C$ is defined as the inverse of~$\Bm A$.  Notice that if $\Bm y$ and~$\Bm z$ are any two vectors  of length~$N-1$ such that~$y_k = 0$, and if $\Bm y^{(k)\!}$ and~$\Bm z^{(k)\!}$ are the same vectors without the $k$-th element, then
$$
\Bm{Ay} = \Bm z\quad\mathbox{implies that}\quad \Bm A^{(k)}\Bm y^{(k)} = \Bm z^{(k)}.\eqmark[reduction]
$$
%This reasoning applied to relation~\ref[ACI] leads to the conclusion that~$C^{[k]}\neq 0$.  If fact, if 
If the element~$C^{[k]}_k$ were equal to zero, then we could apply the just mentioned reasoning to relation~\ref[ACI] obtaining that
$$
\Bm A^{(k)}\,(C^{[k]}_1,\dots,\rlap{\typoscale[1700/]$\backslash$} C^{[k]}_k,\dots, C^{[k]}_N)^T =  \hbox{\bf 0},
$$
and hence that $C^{[k]}_j$ would be zero also for each~$j\neq k$. Since equation~\ref[ACI]  implies that~$\Bm C^{\.[k]}$  cannot be the zero vector, we must conclude that~$C^{[k]}_k\neq 0$.

It is then possible, for each index~$k\in\{1,2,\dots,N\}$, to define the vector
$$
\Bm b^{\lbrace k\rbrace\!}\coloneq \Bm α - {α_k\over C^{[k]}_k}\,\Bm C^{\.[k]},
$$
which, if multiplied by~$\Bm A$, becomes
$$
\eqalign{\Bm A \Bm b^{\lbrace k\rbrace\!} &= \Bm{Aα} - {α_k\over C^{[k]}_k}\,\Bm A \Bm C^{\.[k]} = \Bm f - {α_k\over C^{[k]}_k}\,\Bm I^{\.[k]}.
                                    \cr &= (f_1,\dots,f_{k-1},f_k-{α_k\over C^{[k]}_k}, f_{k+1},\dots,f_N)^T.}
$$
Since the vector~$\Bm b^{\lbrace k\rbrace\!}$ is such that~$b^{\lbrace k \rbrace\!}_k=0$, reasoning~\ref[reduction] applied to this last equation says that
$$
\Bm A\,(b^{\lbrace k\rbrace}_1,\dots,\rlap{$\!$\typoscale[1700/]$\backslash$} b^{\lbrace k\rbrace}_k,\dots, b^{\lbrace k\rbrace}_N)^T= \Bm f\,^{(k)}.
$$
So, the vector that we called~$\Bm β^{\lbrace k\rbrace\!}$ happens to be the vector~$\Bm b^{\lbrace k\rbrace\!}$ without the $k$-th element, i.e.,
$$
\Bm β^{\lbrace k\rbrace\!} = (b^{\lbrace k\rbrace}_1,\dots,\rlap{$\!$\typoscale[1700/]$\backslash$} b^{\lbrace k\rbrace}_k,\dots, b^{\lbrace k\rbrace}_N)^T.
$$

Finally, for each~$k$, the evaluation of~$s^{(k)\!}$ at the point~$x_k\in X$ can be computed in the following way,
$$
s^{(k)\!}(x_k) = \Bm t^{(k)\!}(x_k)^T \Bm β^{\lbrace k\rbrace\!} =  \Bm t(x_k)^T \Bm b^{\lbrace k\rbrace\!} = (\Bm A \Bm b^{\lbrace k\rbrace\!})_k = f_k-{α_k\over C^{[k]}_k},
$$
by reminding that~$\Bm t(x_k)^T\!$ is the $k$-th row of the interpolation matrix~$\Bm A$.~\QED
\postskip

We are now ready to derive an alternative formula also for the previously defined vector~$\Bm U$.  The following calculations are just a continuation of the previous ones, so neither the context nor the used notations will be repeated.
\corollary
The elements~$U_k = \norm{s}_\phi^2-\norm{s^{(k)\!}}_\phi^2$ of the vector~$\Bm U$ can be computed also with the formula
$$
U_k =  α_k\,E_k = {α_k^2\over C_{k,k}}.
$$
\proof
The functions $s$ and~$s^{(k)\!}$ are expressed in terms of the kernel translates as
$$
\eqalign{ s(x) &= \Bm t(x)^T\Bm α \cr
     s^{(k)\!}(x) &= \Bm t^{(k)\!}(x)^T\Bm β^{\lbrace k\rbrace}}
\quad\mathbox{for each $x\in\Omega$.}
$$
This means that the squares of their native space norms assume the values
$$
\eqalign{\norm{s}_\phi^2 &= \Bm α^T\! \Bm A\, \Bm α = \Bm α^T\Bm f \cr
    \norm{s^{(k)\!}}_\phi^2 &= (\.\Bm β^{\lbrace k\rbrace\!})^T\! \Bm A^{(k)}\, \Bm β^{\lbrace k\rbrace} = (\.\Bm β^{\lbrace k\rbrace\!})^T \Bm f\,^{(k)} = (\Bm b^{\lbrace k\rbrace\!})^T \Bm f.
}
$$
By remembering how~$\Bm β^{\lbrace k\rbrace}$ was defined, their difference evaluates to
$$
\eqalign{\norm{s}_\phi^2 -  \norm{s^{(k)\!}}_\phi^2 &= \Bm α^T\Bm f - (\Bm b^{\lbrace k\rbrace\!})^T \Bm f   = (\Bm α - \Bm b^{\lbrace k\rbrace\!})^T \Bm f \cr
                      &= \biggl({α_k \over C^{[k]}_k}\,\Bm C^{\.[k]}\biggr)^T\Bm f ={α_k \over C^{[k]}_k}(\Bm C^{\.[k]})^T\Bm f \cr
                      &= {α_k \over C^{[k]}_k}\,α_k = E_k\,α_k.
}
$$
We also used the fact that the matrix~$\Bm C$ is symmetric (in fact, it is the inverse of the symmetric interpolation matrix~$\Bm A$) and that, by definition, $\Bm{Cf} = \Bm α$.~\QED


Incidentally, just as an additional observation, the combination of the formulas obtained for the elements of $\Bm E$ and~$\Bm U$ with their definitions also gives us alternative expressions for the diagonal elements of the matrix~$\Bm C$ and the interpolation coefficients~$\Bm α$.  In fact,
$$
\left\{\eqalign{ &{α_k\over C_{k,k}}=f_k-s^{(k)\!}(x_k) \cr
                 &{α_k^2\over C_{k,k}} = {(f_k-s^{(k)\!}(x_k))^2\over P_{X^{(k)\!}}^2(x_k)}}\right.
\quad\hbox{implies that}\!\qquad
\left\{\hbyw{6ex}{0em}\right.\!\eqalign{&α_k = {f_k-s^{(k)\!}(x_k) \over P_{X^{(k)\!}}^2(x_k)}\cr
                &C_{k,k} = {1 \over P_{X^{(k)\!}}^2(x_k)}}
$$




%Given the hdfh  return the variables jdjdjd which denotes respectively

It is clear that the computational complexity of the vector~$\Bm U$ and the vector~$\Bm E$ are similar---both vectors can be computed in time~$O(N^3)$, where $N$~is the number of data sites. 
Just to give a concrete example how to actually implement these calculations on a calculator, we present in sequence two MATLAB codes that allow to simultaneously compute both $\Bm E$ and~$\Bm U$, starting from the interpolation matrix~$\Bm A$ and the vector~$\Bm f$ of function values. As a side effect, these codes compute also the matrix~$\Bm C$, the vector of coefficients~$\Bm α$ and the square of the native space norm of~$s$.
The first code uses the standard MATLAB linear system solver (the backslash operator); the second one instead finds the solution of the linear system by first computing the inverse of~$\Bm A$ using the {\tt pinv} algorithm, which provides maximal stability by sacrificing speed---this is what Fassauer~\cite[fasshauer_2007]~does when he computes the cross validation vector~$\Bm E$.

\ttline=-2 % no line numbering
\verbinput (-) MATLAB/loocv_vector.m

\medskip

\verbinput (-) MATLAB/loocv_vector_pinv.m

\postskip


Going now back to its original definition, the vector~$\Bm U\.$ has elements~$U_k = \norm s_\phi^2 - \norm{s^{(k)\!}}_\phi^2$, which satisfy the relation
$$
0 ≤ U_k ≤ \norm s_\phi^2.
$$
Instead of considering the absolute increment of the square of the norm between $s^{(k)\!}$ and~$s$, we can compute the relative increment
$$
Q_k \coloneq {\norm s_\phi^2 - \norm{s^{(k)\!}}_\phi^2 \over \norm s_\phi^2}.
$$
This ratio of course has no meaning if~$\norm s_\phi = 0$.  But in this case it happens that also $\norm{s^{(k)\!}}_\phi = 0$ for all~$k$, therefore it is reasonable to set $Q_k = 0$ as well.

\label[Qdef]
\preskip
\definition
Given a set~$X = \{\range x_N/\}\subset\Omega\subseteq\R^d$ of data sites, with $N ≥ 2$, a set $\Bm F = \{\range f_N/\}$ of function values, and a radial kernel~$\phi$ which is positive definite on~$\R^d$, we define the vector
$$
\Bm Q\coloneq (\range Q_N/)
$$
of relative increments of squared native space norms by setting
$$
\Bm Q\coloneq \cases{\displaystyle{\Bm U \over \norm s_\phi^2} & if $\norm s_\phi\neq 0$ \cr
\hbyw{2.6ex}{.5em}\hbox{\bf 0} & if $\norm s_\phi = 0$}
$$
\postskip


The goal stated at the beginning of this section, to be able to tell whether a sample comes from a regular function or not, can now be achieved.  In fact, it seems like there is enough information stored in the vector~$\Bm Q$ about the regularity of the sample from which it is computed. So, we are tempted to define a regularity classifier as a function of it.

\preskip
\definition
If $X$, $F$, $\phi$ and~$\Bm Q$ are like in definition~\ref[Qdef], and if~$\tau$ is a positive real number, we define the {\em $\phi$-regularity classifier}~$\Cal R$ with {\em tolerance~$\tau$} as the function that maps the pair of data~$(X, F)$ to either the value $1$ or~$-1$ in the following way:
$$
\Cal R(X, F) \coloneq \cases{1 & if $\norm{\Bm Q}_1 ≤ \tau$ \cr
                            -1 & if $\norm{\Bm Q}_1 > \tau$,}
$$
where $\norm{\Bm Q}_1$ is the $1$-norm of~$\Bm Q$ as defined in~\ref[p-norm], which in this case has the simpler expression
$$
\norm{\Bm Q}_1 = \sum_{k=1}^N|Q_k| = \sum_{k=1}^N Q_k.
$$
\postskip


% ought to be  decision value
