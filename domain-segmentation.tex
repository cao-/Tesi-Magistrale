

In this section we present the first phase of a two-phase algorithm to recover discontinuous functions from scattered data; the second part will be disclosed in the following section.  This algorithm is mainly intended to demonstrate the applicability of the regularity classifier described in Chapter~\ref[classificationchap], and must not be viewed as a definitive solution to the problem. Anyhow, it seems to produce satisfactory results. The proposed method greatly takes inspiration from the kernel-based adaptive approximation algorithm outlined by Lenarduzzi and Shaback~\cite[lenarduzzi-shaback_2017]. We restrict our attention to the bidimensional case---even if the algorithm allows for a generalisation in arbitrary dimension---because in higher dimensions the required computational cost quickly becomes prohibitive.  The setting of the problem is similar to that of the cited article.


Assume that a set~$X=\{\range x_N/\}\subset\Omega\subseteq ℝ^2$ of scattered data is given together with an associated set~$F=\{\range f_N/\}$ of function values sampled from an unknown function~$f$.  Assume to know  that $f$ is continuous on certain non-overlapping subdomains~$\Omega_i\subseteq\Omega$, but not globally.  The goal is to recover the function~$f$ on the whole domain~$\Omega=\bigcup_{i}\Omega_i$, starting from its sampled data~$(X, F)$. There will be no a priori assumptions on the form and placement of the singularities.



If a rough direct interpolation is attempted, without taking into account the discontinuities at the boundaries of the subdomains, then the reproduction quality  could be suboptimal--- the Gibbs phenomenon may appear along the boundaries (Fornberg and Flyer~\cite[fornberg-flyer_2012], Jae-Hun~\cite[jae_hun_2007]), and the discontinuity curves won't be outlined well.
One strategy to successfully recover such a discontinuous function is to first identify the different subdomains, or equivalently their boundaries, and then use this acquired knowledge to perform interpolation on each subdomain separately.  The main principle used here is that ``The approximation properties should determine the domains and their boundaries, not the other way round'' (Lenarduzzi and Shaback~\cite[lenarduzzi-shaback_2017]). First we determine the domains by inspecting the local regularity of the function, and then their boundaries as a consequence.
