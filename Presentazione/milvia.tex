\documentclass[xcolor=dvipsnames,9pt,mathserif]{beamer}
\mode<article> {
\usepackage{subfigure}
\usepackage[indention=5pt]{caption}
\captionsetup[subfigure]{margin=5pt, singlelinecheck=false}
% \captionsetup{font=footnotesize}
\usepackage{graphicx}
\usepackage{amssymb}
\usepackage{pifont}
\usepackage{pgf,pgfarrows,pgfnodes,pgfautomata,pgfheaps,pgfshade}
\usepackage{listings,amsmath,multimedia}
\usepackage{pgflibraryarrows}
\usepackage{tikz}
\usepackage{mathtools}
\usepackage{empheq}
\usepackage{latexsym}
\usepackage{subfigure}
\usepackage{graphics}
\usepackage{amssymb}
\usepackage{pifont}
\usepackage{pgf,pgfarrows,pgfnodes,pgfautomata,pgfheaps,pgfshade}
\usepackage{listings,amsmath,multimedia}
\usepackage{pgflibraryarrows}
\usepackage{tikz}
\usepackage{mathtools}
\usepackage{empheq}
\usepackage{latexsym}
\usepackage{subfloat}
\usepackage{enumerate}
\usepackage{relsize}
\usepackage{bigints}
\usepackage[utf 8]{inputenc}
\usepackage{english}
\usepackage[english]{babel}
\usepackage{ragged2e}
\usepackage{amsmath}
\usepackage{amsthm}
\usepackage{amsfonts}
\usepackage{amssymb}
\usepackage{dsfont}
\usepackage{mathrsfs}
\usepackage{graphicx}
\usepackage{setspace}
\usepackage{tikz}
\usepackage{pgfplots}
\usepackage{color}
\usepackage{mathabx}
\usepackage{bm}
\usepackage{cancel}
}
\everymath{\displaystyle}
\setcounter{MaxMatrixCols}{20}
%dvips -Ppdf -G0 milviaMATAA_2017.dvi -o milviaMATAA_2017.ps
%ps2pdf milviaMATAA_2017.ps milviaMATAA_2017.pdf
\parindent=0pt
\newcommand{\pit}{\pause\item}
\newcommand{\vs}{\vskip 0.3truecm}

\newlength{\sepwid}
\newlength{\onecolwid}
\newlength{\twocolwid}
\newlength{\threecolwid}
%\setlength{\paperwidth}{48in} % A0 width: 46.8in
%\setlength{\paperheight}{36in} % A0 height: 33.1in
\setlength{\sepwid}{0.0175\paperwidth} % Separation width (white space) between columns
%\setlength{\onecolwid}{0.31\paperwidth} % Width of one column
%\setlength{\twocolwid}{0.6375\paperwidth} % Width of two columns
%\setlength{\threecolwid}{0.965\paperwidth} % Width of three columns
\setlength{\twocolwid}{0.99\paperwidth}
\setlength{\onecolwid}{0.5\paperwidth}
%\newcommand{\red}[1]{{\color{red} #1 }}     % RS
%\newcommand{\blue}[1]{{\color{blue} #1 }}   %
%\newcommand{\green}[1]{{\color{green} #1 }} %
%\usepackage{mathptmx}      % use Times fonts if available on your TeX system
%\usepackage{amsmath,amsthm}
%\usepackage{amsfonts}
\definecolor{red-}{rgb}{1.0,0.0,0.0}
\definecolor{grey}{rgb}{0.6, 0.6, 0.6}
\definecolor{brown}{rgb}{0.5,0.2,0.0}
\definecolor{brown-}{rgb}{0.0,0.1,1.0}
\definecolor{green-}{rgb}{0.0, 0.6, 0.0}
\definecolor{gold}{rgb}{0.8,0.7,0.0}
\definecolor{black}{rgb}{0.0,0.0,0.0}
\definecolor{DarkGreen}{rgb}{0.0,0.3,0.2}
\definecolor{LightGreen}{rgb}{0.8,1.0, 0.8}
\definecolor{yellow}{rgb}{0.9,0.9,0.0}
\definecolor{blue-}{rgb}{0.0,0.1,1.0}
\definecolor{magenta-}{rgb}{1.0,0.0,1.0}

\def\red#1{{\textcolor{red-}{#1}}}
\def\grey#1{{\textcolor{grey}{#1}}}
\def\green#1{{\textcolor{green-}{#1}}}
\def\dgreen#1{{\textcolor{DarkGreen}{#1}}}
\def\brown#1{{\textcolor{brown-}{#1}}}
\def\gold#1{{\textcolor{gold}{\bf #1}}}
\def\black#1{{\textcolor{black}{\bf #1}}}
\def\brown#1{{\textcolor{brown}{\bf #1}}}
\def\yellow#1{{\textcolor{yellow}{#1}}}
\def\blue#1{{\textcolor{blue-}{#1}}}
\def\magenta#1{{\textcolor{magenta-}{#1}}}


%\def\para{\vspace{2 mm}}




\newcommand\bA{{\bf A}}
\newcommand\bu{{\bf u}}
\newcommand\blambda{{\boldsymbol\lambda}}
\newcommand\RR{\mathbb{{R}}}





\newcommand\ep{\varepsilon^2}
\newcommand\Runo{\sqrt{1+\ep h^2}}
\newcommand\Rdue{\sqrt{1+2\ep h^2}}
\newcommand\Runodue{\sqrt{1+\ep \frac{h^2}{2}}}


\newcommand\xunmezzo{x_{i+\frac12}}
\newcommand\uunmezzo{u_{i+\frac12}}
\newcommand\Stot{{\cal S}_{2k-2}^{i}}
\newcommand\St{{\cal S}_k^{i,b}}
\newcommand\Stb{{\cal S}^{i,b}}
\newcommand\Std{{\cal S}_k^{i,b^*}}
\newcommand\Sts{{\cal S}_k^{i,\bar{b}}}
\newcommand\Ps{P_k^{i,\bar{b}}}
\newcommand\Po{P_k^{i,{b}}}
\newcommand\Pd{P_k^{i,{b^*}}}
\newcommand\Rmq{R^{i,b}_k}
\newcommand\Rmqb{R^{i,b}}
\newcommand\Rmquno{R^{i,1}}
\newcommand\Rmqunob{\bar{R}^{i,1}}
\newcommand\Rmqbb{\bar{R}^{i,b}}
\newcommand\Rmqdue{R^{i,2}}
\newcommand\Rmqdueb{\bar{R}^{i,2}}
\newcommand\Emquno{E^{i,1}}
\newcommand\Emqdue{E^{i,2}}
\newcommand\Emqb{E^{i,b}}
\newcommand\eoptdue{\frac{u''_{i+\frac12}}{u_{i+\frac12}}}
\newcommand\eopttre{-\frac{u_{i+\frac12}'''}{3 u_{i+\frac12}'}}
\usepackage[all]{xy}
\newcommand{\be}{\begin{equation}}
\newcommand{\ee}{\end{equation}}
\newcommand{\bee}{\begin{equation*}}
\newcommand{\eee}{\end{equation*}}
\newcommand\eps{{\varepsilon}}

%%%%%%%%%%%%%%%%%%%%%%%%%%%%%%%%%%%%%%%%%%%%%%%%%%%%%%%%%%%%%%%%%%%%%%%%%%%%%%%%
\newcommand{\coll}[1]{\emph{\color{red} #1}}
\newcommand{\col}[1]{{\bf\color{PineGreen} #1}}
\newcommand{\gn}{\color{JungleGreen}}
\newcommand{\rd}{\color{red}}
%\newcommand{\bl}{\color{blue}}
\newcommand{\mg}{\color{Magenta}}
\newcommand{\gnn}{\color{PineGreen}}
\newcommand{\bl}[1]{{\bf\color{NavyBlue}#1}}
%%%%%%%%%%%%%%%%%%%%%%%%%%%%%%%%%%%%%%%%%%%%%%%%%%%%%%%%%%%%%%%%%%%%%%%%%%%%%%%%%%%%%%
%\newcommand{\nbl}{\color{blue}}
%\newcommand{\nbl}{\color{RedOrange}}

\mode<presentation>{
%\usetheme[left]{Marburg}
%\usetheme{Berkeley}
%\usetheme{PaloAlto}
%\usetheme{Warsaw}
\usetheme{Singaporemio}
%\usecolortheme[rgb={0.3,0.2,0.5}]{structure}
%carini NavyBlue RoyalBlue JungleGreen
\usecolortheme[named=Red]{structure}
%\usecolortheme{sidebartab}
%\usecolortheme[named=ForestGreen]{structure}
%\usecolortheme[named=Emerald]{structure}
%\usecolortheme[named=RedOrange]{structure}
\usefonttheme[]{structurebold}
\useinnertheme[shadow]{rounded}
% GreenYellow, Yellow, Goldenrod, Dandelion, Apricot, Peach, Melon,
%YellowOrange, Orange, BurntOrange, Bittersweet, RedOrange, Mahogany, Maroon, BrickRed, Red,
%OrangeRed, RubineRed, WildStrawberry, Salmon, CarnationPink, Magenta, VioletRed,
%Rhodamine, Mulberry, RedViolet, Fuchsia, Lavender, Thistle, Orchid, DarkOrchid, Purple, Plum, Violet,
%RoyalPurple, BlueViolet, Periwinkle, CadetBlue, CornflowerBlue, MidnightBlue, NavyBlue, RoyalBlue, Blue, %Cerulean, Cyan, ProcessBlue, ,
% SkyBlue, Turquoise, TealBlue, Aquamarine, BlueGreen, Emerald, JungleGreen,
%SeaGreen, ForestGreen, PineGreen, LimeGreen, YellowGreen, SpringGreen, OliveGreen, RawSienna, Sepia,
%Brown, Tan, Gray, Black, White

%colore scritte
%    \setbeamercolor{slidetitle}{white}
%   \setbeamercolor{frametitle}{fg=white,bg=newcolor}

\setbeamertemplate{items}[ball unnumbered]
%\setbeamertemplate{items}[triangle]
% \setbeamercovered{transparent}
% \setbeamertemplate{tableofcontents shaded}[default][50]

\setbeamersize{text margin left=0.15cm,text margin right=0.25cm} }
%%%%%%%%%%%%%%%%%%%%%%%%%%%%%%%%%%%%%%%%%%%%%%%%%%%%%%%%%%%%%%%%%%%%%%%%%%%%%%%%%%%%%%%%%%%%%%%%%%%%%%


\title[]
{{{Recovering discontinuous functions via  local multiquadric
interpolation with adaptive nonlinear estimate
of the shape parameter}}}

\author[]{{\bf{Milvia Rossini}}, Francesc Ar\`andiga, Rosa Donat, Lucia Romani }

\vskip 0.2truecm
%\institute[Universit\`{a}  di Milano - Bicocca]{\includegraphics[width=10mm]{logo_unimib.eps} \\ Universit\`{a}di %Milano - Bicocca}
\institute{
%%\large University of Bologna, Italy\\
\includegraphics[width=15mm]{logo_unimib.png} \\
\vskip+1mm
{University of Milano - Bicocca}\\
\vskip 0.1truecm
\vskip 0.1truecm
{University of Valencia -- University of Bologna}}


\date[]{NUMTA 2019}
%\logo{\includegraphics[width=10mm]{logos_unimib.eps}}
\begin{document}
\begin{frame}{}
  \titlepage
\end{frame}

\part<presentation>{Main Talk}

\section[Introduction]{Introduction}
\subsection[]{}

\begin{frame}
\frametitle{Introduction}
\begin{itemize}
%It is a well-known fact that any global or high-order
%approximation method suffers from the Gibbs phenomenon if the function
%to be approximated has a jump discontinuity in the given domain. Thus,
\item The faithful recovery of a discontinuous function is a \coll{challenging} problem
%\pause
\vskip 0.3truecm
\item Wide literature on the topic also by \textcolor{red}{RITA} people [RITA $\&$ co]
%\pause
\vskip 0.3truecm
\item More efforts are still needed to construct robust Radial Basis Function (RBF) approximants for functions with jump discontinuities
%\pause
\vskip 0.3truecm
\item Effective strategy: \coll{ variably scaled kernels} that
incorporates discontinuous scaling functions at or near the jump
points.
%\pause
\vskip 0.3truecm
\item Advantage: it produces a true discontinuous interpolant and
then a faithful recovery of the unknown function.
%\pause
\vskip 0.3truecm
\item This approach requires to know or to estimate in
advance the positions of the discontinuity points  and an edge
detection strategy is needed as preliminary step \cite{erb}.
\end{itemize}
\end{frame}
\begin{frame}
\frametitle{Introduction}

\begin{itemize}
\item   \coll{E}ssentially \coll{N}on-\coll{O}scillatory (\coll{ENO}) and \coll{W}eighted ENO
(\coll{WENO}) polynomial techniques do
not require a preliminary edge detection step
\item  They allow to reduce oscillations near discontinuities by the use of
suitable data dependent \coll{smoothness indicators}
%\pause
\item in \cite{GJ17a} and \cite{GJ17b}, polynomials have been replaced by \coll{multiquadrics}  in adaptive finite difference  ENO/WENO schemes for hyperbolic
conservation laws
%\pause
\item it is \coll{numerically} observed
that, by locally optimizing the \coll{shape parameter} of the multiquadric
interpolant, the accuracy of the RBF-WENO scheme is enhanced
in \coll{smooth} regions, with respect to the polynomial WENO method
%\pause
\vskip 0.3truecm
\item their
RBF-WENO reconstruction \coll{is not appropriate} in the vicinity of a jump
discontinuity
%\pause
\vskip 0.3truecm
\item hence they need to
 revert to classical polynomial-based WENO reconstructions when close to
 shocks or contact discontinuities.
\end{itemize}
\end{frame}
\begin{frame}
\frametitle{Motivations }

\begin{enumerate}
\item The desire   to fully understand from a theoretical point of view the
computational results shown in \cite{GJ17a}, \cite{GJ17b} and to understand the need of reverting to polynomial WENO near jumps;
%\pause
\vskip 0.6truecm

\item to propose \coll{new true} WENO multiquadric techiques;

\vskip 0.6truecm
%\pause
\item to study
how  \coll{WENO adaptivity} may be incorporated in a
MQ-framework to estimate the  optimal shape parameter that allows to improve/preserve  accuracy
both in smooth regions and in the vicinity of jump discontinuities.

\end{enumerate}
\end{frame}
\section{ENO/WENO local polynomial interpolation}
\subsection{}
\begin{frame}
\frametitle{Notations}
We consider
\begin{itemize}
\item $b,k\in \mathbb{{N}},$  $k\ge 2$, $b\ge 1,$ and $k$  values $u_j$, $j=i-k+b+1, \ldots, i+b $ of
$$u: \Omega{\,\subset\RR} \rightarrow \RR$$
 at equally spaced
points $x_j \in \Omega$ with step-size $h$.
\item the midpoint of the interval $[x_i,x_{i+1}],$ $\xunmezzo:=\frac{x_i+x_{i+1}}{2}$
\item  the
  stencils of $k$ points containing $x_i$ and $x_{i+1}$
$$\St= \{ x_{i-k+b+1}, \ldots,x_{i+ b}\},\quad1 \leq b \leq k-1$$
\end{itemize}
%\pause
Our aim is to provide
\coll{accurate predictions} to $\uunmezzo:=u(\xunmezzo).$

\vskip 0.2truecm
%\pause
\small{\begin{itemize}\item The term \coll{smooth} will  refer to the function $u$ being as smooth as required by the theoretical facts that we are going to consider.
\item
The term \coll{discontinuity} will refer to an isolated discontinuity in an interval adjacent to $[x_i,x_{i+1}].$\end{itemize}}
  \end{frame}
  \begin{frame}
  \frametitle{ENO local polynomial
  interpolation}


\begin{enumerate}
\item  For each
stencil $\St$  compute the undivided   differences  $$
  \left[\St\right]:=h^{k-1}u[x_{i-k+b+1},\ldots,x_{i+ b}]$$
%\pause
\begin{itemize}
  \item if  $\St$ is in a
smooth region $\Rightarrow$
$
\left[\St\right]=O({h^{k-1}}) $
%\pause
\item if $u$ has a discontinuity
 in { $(x_{i-k+b+1}, x_{i+b})$} $\Rightarrow$ $\left[\St\right]=O(1)$
 \end{itemize}
 %\pause
 \item  select the stencil
\[
\left| \left[\Sts\right] \right| = \min_{b=1,\dots,k-1} \left| \left[\St\right] \right|,\]
and define
\[P^{ENO}_k(x):=\Ps(x).\]
\end{enumerate}
%\pause
\begin{block}{}
If $u$ is smooth in the convex hull of  $\Sts$
$$
\uunmezzo=P^{ENO}_k(\xunmezzo)+O(h^k).
$$
If $u$ has  a
discontinuity at $[x_i,x_{i+1}]$, the accuracy  is lost, but $ P^{ENO}_k(x)$
is \coll{monotone} in $[x_i,x_{i+1}]\Rightarrow$ \coll{no oscillations}.
\end{block}

  \end{frame}
  \begin{frame}
  \frametitle{WENO local polynomial interpolation}
  $${P}^{WENO}_{2k-2}(\xunmezzo) = \sum_{b=1}^{k-1}\omega^{i,b}_k \Po(\xunmezzo)
$$
where
$$
 \omega_k^{i,b} \ge 0, \,\, \quad \sum_{b=1}^{k-1} \omega_k^{i,b}=1
$$

  are chosen to satisfy the following properties:
  \vskip 0.2truecm

\begin{itemize}
\item $\omega_k^{i,b} \approx 0$ if the stencil $\St$ crosses a
  discontinuity so that discontinuous stencils have essentially no
  contribution to the convex combination;
\vskip 0.2truecm

\item $\omega_k^{i,b}$  near to some  optimal weights otherwise.
\end{itemize}
\end{frame}
  \begin{frame}
  \frametitle{WENO weights}

$$
\omega_k^{i,b}=\frac{\alpha_k^{i,b}}{\sum_{\ell=1}^{k-1} \alpha^{i,\ell}_k }
\quad
\hbox{with}
\quad
{\alpha_k^{i,b}=\frac{C_k^b}{(\epsilon +I_k^{i,b})^{\rho}},}
$$
where
\small{
\begin{itemize}
\item $C_k^b\, |$ $C_k^b\ge0,$   $\sum_{b=1}^{k-1} C_k^b=1,\,\,$  and $\,\,\uunmezzo =\sum_{b=1}^{k-1}C^b_k
 \Po(\xunmezzo)   + O(h^{2k-2})$
%\pause
\item $\epsilon$  avoids null
  denominator and $\rho\in\mathbb{N}$ (usually   $\epsilon=h^2$  and $\rho\geq 2$)
%\pause
\item $I_k^{i,b}$ are \coll{smoothness indicators}  such that
\begin{itemize}
\item if all the stencils $\St$ are in a smooth region $\Rightarrow$ $\omega^{i,b}_k=C^b_k+O(h^{k-1}),$ and
      $$\uunmezzo={P}^{WENO}_{2k-2}(\xunmezzo) +O(h^{2k-2})$$
\item if $u$ is smooth in $[x_i,x_{i+1}]$ and  there is a discontinuity in  {$(x_{i-k+2},x_{i+k-1})$}, at least one stencil  \coll{ does not cross } the singularity and the approximation order is that of the corresponding \coll{ENO} polynomial
    $$
\uunmezzo={P}^{WENO}_{2k-2}(\xunmezzo) +O(h^{k})$$
\end{itemize}
\end{itemize}
%\pause
If there is a discontinuity in
$[x_i,x_{i+1}]$, the WENO construction looses all accuracy but, as in
the ENO case, it does
not create spurious oscillations.}
\end{frame}
\section{WENO MQ-RBF local interpolation: $k=3,$ $b=2$}
\subsection{}
\begin{frame}\frametitle{Local multiquadric RBF 3-point interpolation}

Polynomials are substituted by   MQ-
interpolants  based on the 3-point stencils ${\cal
  S}_3^{i,b}$
$$
\Rmqb_3(x,\varepsilon)=\lambda_1 \phi(|x-x_{i+b-2}|,\varepsilon)
+ \lambda_2 \phi(|x-x_{i+b-1}|,\varepsilon)
+ \lambda_3 \phi(|x-x_{i+b}|,\varepsilon),
$$
where $\phi(y,\varepsilon)= \sqrt{1+\ep  y^2
 }.$

The coefficients depend on {$ \phi(h,\varepsilon)$ and $\phi(2h,\varepsilon).$ Their expansion by Taylor series, } gives

$$
\Rmqb_3(x_{i+\frac12},\varepsilon)=\Rmqbb_3(x_{i+\frac12},\varepsilon)
+ O(h^6)\quad b=1,2
$$
with
%\pause
\begin{eqnarray*}
\Rmqbb_3(x_{i+\frac12},\varepsilon)
&=&\left(
\frac{27}{1024}\,{\eps}^4\,h^4-\frac{1}{8}
\right)u_{i+b-2}
+\left(
%    -\frac{795\,{\eps}^6\,h^6}{1024}+
\frac{171}{512}\,{\eps}^4\,h^4-\frac{3}{16}\,{\eps}^2\,h^2+\frac{3}{4}
\right)u_{i+b-1}\\
&+&\left(
-\frac{441}{1024}\,{\eps}^4\,h^4+\frac{3}{16}\,{\eps}^2\,h^2+\frac{3}{8}
 \right)u_{i+b}.
\end{eqnarray*}

%\pause
Then, the approximation error is
 $$\Emqb_3(x_{i+\frac12},\varepsilon)=\uunmezzo-\Rmqbb_3(x_{i+\frac12
},\varepsilon) + O( h^6), \quad b=1,2.$$

\end{frame}
\begin{frame}\frametitle{Local multiquadric RBF 3-point interpolation}
Expanding $\{u_{i+b-2},u_{i+b-1},u_{i+b}\}$
 by %evaluating $u_{i+b-2},$ $u_{i+b-1}$ and $u_{i+b}$, $b=1,2$,
         %by
Taylor series  centered at $\xunmezzo,$ we get

$$
\Emqb_3(x_{i+\frac12},\varepsilon)= \frac{1}{16}(3 u_{i+\frac12}' \varepsilon^2 + u_{i+\frac12}''') h^3
-\frac{3}{128}(3u_{i+\frac12}\varepsilon^4+u^{(iv)}_{i+\frac12}) h^4+
O( h^5),  \quad b=1,2.
$$

Hence
%The approximation order is improved by choosing
$$\varepsilon_{\text{opt}}^2 := -\frac{u_{i+\frac12}'''}{3 u_{i+\frac12}'}
%=- \frac{-u_{i-1}+3u_i-3u_{i+1}+u_{i+2}}{3(u_{i+1}-u_i)}\]
\quad \Rightarrow \quad
\Emqb_3(x_{i+\frac12},\varepsilon_{\text{opt}})=O(h^4),  \quad b=1,2,
$$
%In this case,  only a third order local approximation of $\uunmezzo$
%can be computed via the formulas (\ref{eq:aprad31})  and
%(\ref{eq:aprad3r}). Again, third order
and a fourth-order approximation can be directly computed using $\Rmqbb_3(x_{i+\frac12
},\varepsilon).$

\vskip 0.3truecm
%\pause
In practice, \coll{the optimal  value} of $\varepsilon^2_{\text{opt}}$ needs to be
approximated.%, since $u$ is unknown.
\vskip 0.3truecm
An approximation of  $\varepsilon_{\text{opt}}^2$ such that
$$
 \varepsilon^2=\varepsilon_{\text{opt}}^2 + O(h^p)\quad \Rightarrow \quad \Emqb_3(\xunmezzo,\varepsilon)=O(h^{\min\{p+3, 4\}}), \quad b=1,2.
$$
\end{frame}
\begin{frame}\frametitle{Local multiquadric RBF 3-point interpolation}
\begin{block}{Proposition}
Let  $u$ \coll{smooth} in $([x_{i-1},x_{i+2}])$, then

$$
\bar{u}_{i+\frac12}^{'''}:=\frac{-u_{i-1}+3u_i-3u_{i+1}+u_{i+2}}{h^3}=\uunmezzo'''
+ O(h^2)$$ $$ \bar{u}_{i+\frac12}^{'}:=\frac{u_{i+1}-u_i}{h}=\uunmezzo'+O(h^2),$$
 hence

$$
{{\varepsilon}_{lin}^2}:=-\frac{\bar{u}_{i+\frac12}^{'''}}{3 \bar{u}^{'}_{i+\frac12}}=
{\varepsilon^2_{\text{opt}} + O(h^2) \quad\Rightarrow\quad}
\Emqb_3(x_{i+\frac12},\varepsilon_{{lin}})=O(h^4).$$
\end{block}
\end{frame}
\begin{frame}\frametitle{Local multiquadric RBF 3-point interpolation}
\begin{block}{Proposition}
%But if The unknown function has a discontinuity in the first of last interval of the considered stencils, the accuracy is lost.

If $u$ \coll{has a discontinuity} in $[x_{i-1},x_i]$ (or in $[x_{i+1},x_{i+2}]$)
then

$$
{\varepsilon_{lin}^2= O\left(\frac{1}{h^3}\right)},
\quad\Rightarrow\quad
\Emqb_3(x_{i+\frac12},\varepsilon_{\text{lin}})= O\left(\frac{1}{h^2}\right)
\quad { \text{for} }\,\,  b=1  \quad ({\text{or} }\,\, b=2).
$$
\end{block}
\begin{itemize}
\vskip 0.3truecm

%\pause
\item
If we plug this estimate in a WENO technique the accuracy in intervals adjacent to a discontinuity is lost.

%\pause
\vskip 0.3truecm

\item This is the reason of switching to  polynomial WENO near non smooth regions, case that corresponds to the choice $\varepsilon=0.$\end{itemize}
     \end{frame}
     \begin{frame}
     \frametitle{Local MQ-WENO interpolants}
     \begin{eqnarray*}
     R^{WENO}_{4}(x_{i+\frac12})& = &\sum_{b=1}^{2}\omega^{i,b}_3 \Rmqbb_{3}(x_{i+\frac12})\\
%\pause
&=& \omega_3^{i,1} \, \left(\frac{27}{1024}\,{\eps}^4\,h^4-\frac{1}{8}\right) \, u_{i-1}+ \omega_3^{i,1} \, \left(\frac{171}{512}\,{\eps}^4\,h^4-\frac{3}{16}\,{\eps}^2\,h^2+\frac{3}{4}\right)\,u_i\\
&+& \omega_3^{i,2} \, \left(-\frac{441}{1024}\,{\eps}^4\,h^4+\frac{3}{16}\,{\eps}^2\,h^2+\frac{3}{8}\right)  \, u_i\\
&+& \omega_3^{i,1} \, \left(-\frac{441}{1024}\,{\eps}^4\,h^4+\frac{3}{16}\,{\eps}^2\,h^2+\frac{3}{8}\right)\,u_{i+1}\\
&+& \omega_3^{i,2} \, \left(\frac{171}{512}\,{\eps}^4\,h^4-\frac{3}{16}\,{\eps}^2\,h^2+\frac{3}{4}\right) \, u_{i+1}\\
&+&\omega_3^{i,2} \, \left(\frac{27}{1024}\,{\eps}^4\,h^4-\frac{1}{8}\right) \, u_{i+2}.
\end{eqnarray*}
\end{frame}
\begin{frame}
     \frametitle{Local MQ-WENO interpolants}

$$\omega_{3}^{i,b}=\frac{\alpha_{3}^{i,b}}{\sum_{\ell=1}^{2} \alpha^{i,\ell}_{3} },
\quad
\alpha_{3}^{i,b}={ \frac{1/2}{(h^2 +I_3^{i,b})^{2}} }, \ \ b=1,2,
$$
where the smoothness indicators $I_3^{i,b}$, $b=1,2$
\begin{eqnarray*}
I_3^{i,1}&=&\frac{13}{12} (u_{i-1}-2u_{i}+u_{i+1})^2+
\frac{1}{4}(u_{i-1}-4u_{i}+3u_{i+1})^2,
\\
I_3^{i,2}&=&\frac{13}{12}(u_{i}-2u_{i+1}+u_{i+2})^2
+\frac{1}{4}(u_{i+2}-u_{i})^2
\end{eqnarray*}
guarantee that

\begin{enumerate}
\item If $u$ is smooth in $([x_{i-1}, x_{i+2}])$,
$$\omega^{i,b}_3=\frac{1}{2}+O({h^2}), b=1,2;$$
\item While if  $u$ has a discontinuity at $[x_{i+1}, x_{i+2}]$ (or
  $[x_{i-1}, x_{i}]$)
  $$
  \omega^{i,1}_3=1+O({h^4})\quad \omega^{i,2}_3=O({h^4}) $$
(or $\omega^{i,1}_3=O({h^4})$, $\omega^{i,2}_3=1+O({h^4})$)

\end{enumerate}
     \end{frame}
     \begin{frame}

     \frametitle{Local MQ-WENO interpolants}
If $u$ is smooth in $ [x_{i-1}, x_{i+2}]$,  evaluating $u_{i-1},$ $u_i$, $u_{i+1}$ and $u_{i+2}$ by Taylor expansions centered at $\xunmezzo,$ provides
\begin{eqnarray*}
R^{WENO}_{4}(x_{i+\frac12})
&= &(\omega_3^{i,1}+\omega_3^{i,2})\uunmezzo- \frac{1}{16}(3 u_{i+\frac12}' \varepsilon^2 + u_{i+\frac12}''')(\omega_3^{i,1}-\omega_3^{i,2})h^3\\
&-&\frac{3}{128}(3u_{i+\frac12}\varepsilon^4+u^{(iv)}_{i+\frac12})(\omega_3^{i,1}+\omega_3^{i,2}) h^4+ O(h^5), \end{eqnarray*}

%\pause
\begin{block}{Proposition}
If {$u$ is smooth in $([x_{i-1},x_{i+2}])$} then the weight are approximatively equal to one half and then
$$\uunmezzo - R^{WENO}_{4}(x_{i+\frac12},\varepsilon) =
%u_{i+\frac12} - R^{WENO}_{4}(x_{i+\frac12})   =
 \frac{3}{128}(u^{(iv)}_{i+\frac12} + 3 \, \eps^4 \, u_{i+\frac12}) \, h^4+ O(h^5).
$$
\end{block}
%\pause
\begin{block}{Proposition}
If {$u$}
  and has
a jump discontinuity at { $(x_{i+1},x_{i+2}]$} (or at { $[x_{i-1},x_{i})$}) then the weights  will tend to select the 3-point MQ-RBF interpolant based on the stencil  that does not
cross the  discontinuity and the
leading order of the error in the convex combination corresponds to
that of the selected  3-point MQ-RBF interpolant
$$\uunmezzo- R_4^{WENO}(x_{i+\frac12},\varepsilon)
=
\frac{1}{16}(3 u'_{i+\frac12} \varepsilon^2 + u'''_{i+\frac12}) h^3
-\frac{3}{128}(3u_{i+\frac12}\varepsilon^4+u^{(iv)}_{i+\frac12}) h^4+  O(h^4).
% \nonumber
%O(h^3).
$$
\end{block}
     \end{frame}
     \begin{frame}
     \frametitle{Enhancing accuracy in non smooth regions vs smooth regions  }

\begin{block}{Method 1}
The choice
$$\varepsilon_{\text{opt}}^2 = -\frac{u_{i+\frac12}'''}{3 u_{i+\frac12}'} $$
improves the accuracy in regions adjacent to a discontinuity.
\end{block}
%\pause
    \begin{block}{Method 2}
The choice
\[ \varepsilon_{\text{opt}}^4 = -\frac{u^{(iv)}_{i+\frac12}}{3u_{i+\frac12}}\]
improves the accuracy in smooth regions.
\end{block}
%\pause
In \coll{both cases} an estimate of the shape parameter the preserves the improvement is obtained via a \coll{WENO-like adaptive} procedure.
\end{frame}
\begin{frame}
\frametitle{Method 1: Weno-like adaptive estimation of $\varepsilon_{\text{opt}}$  }
 We    consider one-sided first-order estimates
of the third derivatives at the mid point and appropriate WENO-type weights:
\begin{eqnarray*}
%\hat
{u}^{'''}_{i+\frac12,L}& = &
 \frac{1}{h^3}( -u_{i-2} + 3 \, u_{i-1} -3 \, u_{i} +  \, u_{i+1})\,\,
\\
{u}^{'''}_{i+\frac12,R} &= &
 \frac{1}{h^3}( -u_{i} +3 \, u_{i+1} -3 \, u_{i+2} + \, u_{i+3} ),
%= u^{'''}_{i+\frac12}+ O(h),
\end{eqnarray*}

$${
\varepsilon^2_{wen} := -
\frac{\omega^{i,1}_4 {u}^{'''}_{i+\frac12 }
+ {\omega^{i,3}_4} {u}^{'''}_{i+\frac12 }} {3(u_{i+1}-u_{i})/h}
},
$$
where
$${
\omega_{4}^{i,b}=\frac{\alpha_{4}^{i,b}}{ \alpha^{i,1}_{4}+ \alpha^{i,3}_{4} },
\quad
\alpha_{4}^{i,b}={ \frac{1/2}{(h^2 +I_4^{i,b})^{3}} }, \ \ b=1,3
}
$$
with
$$I_4^{i,b}=
( - \, u_{i+b-3} + 3 \, u_{i+b-2} -3 \, u_{i+b-1} +  \, u_{i+b})^2,\quad b=1,3. $$
\end{frame}
\begin{frame}
\frametitle{Method 1: Weno-like adaptive estimation of $\varepsilon_{\text{opt}}$  }

\begin{block}{Proposition}
\begin{enumerate}
\item If {$u$ is smooth $[x_{i-1},x_{i+3}]$} or in $[x_{i-2},x_{i+2}]$ then
\[ \varepsilon^2_{wen} =
\varepsilon_{\text{opt}}^2 +O(h) \quad \Rightarrow \quad \uunmezzo-
R_4^{WENO}(\xunmezzo,\varepsilon_{wen})=O(h^4).\]
%\pause

\vskip 0.4truecm
\item If $u$ has a  discontinuity in $[x_{i-1},x_{i}]$
(or in  $[x_{i+1},x_{i+2}]$)
and it is smooth %\textcolor{blue}
{in $[x_i,x_{i+3}]$ (or in
  $[x_{i-2},x_{i+1}]$)}, then
$$\varepsilon_{wen}^2=\varepsilon_{\text{opt}}^2+ O(h) \quad \Rightarrow \quad
\uunmezzo- R_4^{WENO}(\xunmezzo,\varepsilon_{wen})= O(h^4).$$
\end{enumerate}
\end{block}
     \end{frame}
     \begin{frame}
     \frametitle{Method 2: enhancing accuracy in smooth regions}
 In smooth regions, the chpoice
\[ \varepsilon_{\text{opt}}^4 := -\frac{u^{(iv)}_{i+\frac12}}{3u_{i+\frac12}}.
%\quad (\mbox{or } \hat{\varepsilon}_{\text{opt}}^2 := \frac{u''_{i}}{u_{i}})
\]
provides a \coll{fifth} order accuracy.
%\pause
\begin{block}{Remark}
$$\omega^{i,1}_3=\omega^{i,2}_3=\frac12 +
O(h^2),$$ then the
the term containing $\varepsilon^2 $ in $R^{WENO}_{4}(x_{i+\frac12})$ is
$$
\frac{3}{16}\,(u_{i+1}-u_i)(\omega_3^{i,1} -
\omega_3^{i,2})\,{\eps}^2\,h^2=O(h^5)
$$
and it can be neglected as it is of the order of the error. \end{block}
 \end{frame}

\begin{frame}
\frametitle{A modified MQ-WENO local interpolation}

This observation leads us
to consider the modified multiquadric WENO %based
reconstruction technique

$$
G_4^{WENO}(\xunmezzo,\varepsilon):=\sum_{b=1}^{2}\omega^{i,b}_3
G_{3}^{i,b} (x_{i+\frac12},\varepsilon)
$$
where

$$
G_{3}^{i,1}(\xunmezzo,\varepsilon)
=\left(
\frac{27}{1024}{\eps}^4\,h^4-\frac{1}{8}
\right)u_{i-1}
+\left(
\frac{171}{512}{\eps}^4\,h^4+\frac{3}{4}
\right)u_{i}
+\left(
-\frac{441}{1024}{\eps}^4\,h^4+\frac{3}{8}
 \right)u_{i+1}
 $$

$$
G_{3}^{i,2} (\xunmezzo,\varepsilon)
=\left(
\frac{27}{1024}{\eps}^4\,h^4-\frac{1}{8}
\right)u_{i+2}
+\left(
\frac{171}{512}{\eps}^4\,h^4+\frac{3}{4}
\right)u_{i+1}
+\left(
-\frac{441}{1024}{\eps}^4\,h^4+\frac{3}{8}
 \right)u_{i}.
$$
\end{frame}
\begin{frame}
\frametitle{A modified MQ-WENO local interpolation}

The computations are carried on  via  a \coll{Weno-like adaptive} estimation of $\varepsilon_{\text{opt}}$ by combining one-side estimates of ${u^{(iv)}_{i+\frac12}}$ that  allows to \coll{preserve}  order three of accuracy in the proximity of a discontinuity.

%\pause
\begin{block}{Proposition}
\begin{enumerate}
\item If $u$ is smooth in $[x_{i-3},x_{i+2}]$ or in $[x_{i-1},x_{i+4}]$ then
\[ \varepsilon^2_{wen} =\varepsilon_{\text{opt}}^2 +O(h)  \quad
\Rightarrow \quad \uunmezzo-
G_4^{WENO}(\xunmezzo,\varepsilon_{wen})=O(h^5); \]
%\pause
\vskip 0.4truecm
\item if $u$ has a discontinuity at $[x_{i-2},x_{i}]$ (or $[x_{i+1},x_{i+3}]$), then
$$\varepsilon_{wen}^2=
\varepsilon_{\text{opt}}^2+ O(h) \quad \Rightarrow \quad
\uunmezzo- G_4^{WENO}(\xunmezzo,\varepsilon_{wen})= O(h^3).$$
\end{enumerate}
\end{block}
\end{frame}
\begin{frame}
\frametitle{Summary: orders of accuracy}

\begin{table}
\begin{center}
{$
\begin{tabular}{|c|c|c|c|c|}
\hline \\[-8pt]
& \multicolumn{2}{|c|}{$\uunmezzo - R^{WENO}_{4}(x_{i+\frac12},
\varepsilon$)
}
& \multicolumn{2}{|c|}{$\uunmezzo -G^{WENO}_{4}(x_{i+\frac12},
\varepsilon)$
}
\\
\cline{2-5}
&    $smooth$ & $disc $&  $smooth$ & $disc $\\
\hline
$\varepsilon_{\text{opt}}$ &  $O(h^4)$& $O(h^4)$& $O(h^5)$& $O(h^3)$\\ \hline
$\varepsilon_{lin}$ &  $O(h^4)$& $O(h^{-2})$& $O(h^5)$& $O(1)$\\ \hline
$\varepsilon_{wen}$ & $O(h^4)$& $O(h^4)$& $O(h^5)$& $O(h^3)$\\ \hline
\end{tabular}
$}
 \end{center}
 \vskip 0.4truecm
 \caption{Orders of accuracy for different reconstructions using stencils of $k=3$ points and different
$\varepsilon$. The term
{\em smooth} refers to the function $u$ being as
  smooth as required by the theoretical Propositions. The term {\em disc}
  refers to the function being
 smooth in $[x_{i},x_{i+1}]$, with an
  isolated discontinuity in an interval adjacent to $[x_i,x_{i+1}]$.
}
  \end{table}


\end{frame}
\section{Numerical experiments}
\subsection{}
\begin{frame}
\frametitle{Numerical evidence of approximation orders}
W consider and dyadic samples of increasing size
$$\{ u^{\ell}_i=u( x_i^{\ell}) \}_{i=0, \ldots, 2^{\ell}}\quad {\ell}=6, 7, \ldots, 13, \quad x_i^{\ell}=\frac{i}{2^{\ell}}$$
where
$$u(x)=\left\{ \begin{array}{ccc}
e^{(x-0.5)} & \mbox{ if } & 0 \leq x \leq 0.5,
\\
1+e^{(x-0.5)} & \mbox{ if } & 0.5 < x \leq 1.
\end{array}
 \right.
$$

Let $\Omega_d\subset [0,1]$, and consider the maximum absolute error in such sub interval
$$
e_{\ell}:=\max \left \{ |u(x^{{\ell}}_{i+\frac12}) - \hat{u}^{{\ell}+1}_{2i+1}|, \ x^{\ell}_{i+\frac12}=\frac{x_i^{\ell}+x_{i+1}^{\ell}}{2} \in \Omega_d \right \}
$$

\end{frame}
\begin{frame}
\frametitle{Smooth regions:  far from jump discontinuities $R_4^{WENO}$}
\begin{table}[h!]
\begin{center}
%\scalebox{0.8}
{$
\begin{tabular}{|c|c|c|c|c|}
\hline
& \multicolumn{4}{|c|}{ \  \ $\Omega_{d}=[0,0.25]$} \\
%\hline
\cline{2-5}
${\ell}$ & \multicolumn{2}{|c|}{$\uunmezzo - R^{WENO}_4(x^{\ell}_{i+\frac12},
\varepsilon_{lin})$}
& \multicolumn{2}{|c|}{$\uunmezzo - R^{WENO}_4(x^{\ell}_{i+\frac12},
\varepsilon_{wen})$}
\\
\cline{2-5}
%${\ell}$
&  $e_{\ell}$ & $\log_2(e_{{\ell}-1}/e_{{\ell}}) $
 &$e_{\ell}$ & $\log_2(e_{{\ell}-1}/e_{{\ell}}) $\\
\hline
   6 & 1.4394e-09 & 3.9944e+00  & 1.4394e-09 & 3.9944e+00   \\
   7 & 9.0311e-11 & 3.9972e+00  & 9.0313e-11 & 3.9972e+00   \\
   8 & 5.6555e-12 & 3.9986e+00  & 5.6555e-12 & 3.9986e+00   \\
   9 & 3.5381e-13 & 3.9993e+00  & 3.5381e-13 & 3.9993e+00   \\
   10 & 2.2124e-14 & 3.9996e+00  & 2.2124e-14 & 3.9996e+00   \\
   11 & 1.3831e-15 & 3.9998e+00  & 1.3831e-15 & 3.9998e+00   \\
   12 & 8.6454e-17 & 3.9999e+00  & 8.6454e-17 & 3.9999e+00   \\
   13 & 5.4037e-18 & 4.0000e+00  & 5.4037e-18 & 4.0000e+00   \\
\hline
\end{tabular}
$}
 \end{center}
 \caption{Smooth region, far from jumps: Errors $e_{\ell}$ and estimates of the approximation order
   for    $\hat{u}^{{\ell}+1}_{2i+1}=R_4^{WENO}(x^{\ell}_{i+\frac12},\varepsilon)$
   with %parameter estimation
$\varepsilon^2_{lin}$  and
 $\varepsilon^2_{wen}.$
}
  \end{table}
\end{frame}
\begin{frame}
\frametitle{Smooth regions:  far from jump discontinuities $G_4^{WENO}$}
\begin{table}[h!]
\begin{center}
%\scalebox{1.2}
{$
\begin{tabular}{|c|c|c|c|c|}
\hline
& \multicolumn{4}{|c|}{ \ $\Omega_{d}=[0,0.25]$} \\
%\hline
\cline{2-5}
${\ell}$ & \multicolumn{2}{|c|}{$\uunmezzo -G^{WENO}_4(x^{\ell}_{i+\frac12}, \varepsilon_{lin})$}  &
 \multicolumn{2}{|c|}{$\uunmezzo -G^{WENO}_4(x^{\ell}_{i+\frac12}, \varepsilon_{wen})$}
\\
\cline{2-5}
%${\ell}$
&  $e_{\ell}$ & $\log_2(e_{{\ell}-1}/e_{{\ell}}) $
 &$e_{\ell}$ & $\log_2(e_{{\ell}-1}/e_{{\ell}}) $\\
\hline
  6 & 1.7380e-11 & 5.0110e+00  & 1.7644e-11 & 5.0217e+00   \\
   7 & 5.3900e-13 & 5.0056e+00 & 5.4313e-13 & 5.0110e+00   \\
   8 & 1.6779e-14 & 5.0028e+00  & 1.6844e-14 & 5.0056e+00   \\
   9 & 5.2332e-16 & 5.0014e+00  & 5.2433e-16 & 5.0028e+00   \\
   10 & 1.6338e-17 & 5.0007e+00  & 1.6354e-17 & 5.0014e+00   \\
   11 & 5.1031e-19 & 5.0004e+00  & 5.1056e-19 & 5.0007e+00   \\
   12 & 1.5943e-20 & 5.0002e+00  & 1.5947e-20 & 5.0004e+00   \\
   13 & 4.9817e-22 & 5.0001e+00  & 4.9823e-22 & 5.0002e+00   \\
\hline
\end{tabular}
$}
 \end{center}
 \caption{Smooth region, far from jumps: Errors $e_{\ell}$ and estimates of the approximation order for $\hat{u}^{{\ell}+1}_{2i+1}=G^{WENO}_4(x^{\ell}_{i+\frac12},\varepsilon)$ with parameter estimation
$\varepsilon^4_{lin}$  and
 $\varepsilon^4_{wen}.$
}
%\label{tab:r3b_smo}
  \end{table}

\end{frame}
\begin{frame}
\frametitle{Smooth regions: next to a jump discontinuity $R_4^{WENO} $}
\begin{table}[h!]
\begin{center}
%\scalebox{0.8}
{$
\begin{tabular}{|c|c|c|c|c|}
\hline
& \multicolumn{4}{|c|}{ \ $\Omega_{d}=[0,0.5]$} \\
%\hline
\cline{2-5}
${\ell}$ & \multicolumn{2}{|c|}{$\uunmezzo -R^{WENO}_4(x^{\ell}_{i+\frac12},
\varepsilon_{lin})$}
& \multicolumn{2}{|c|}{$\uunmezzo -R^{WENO}_4(x^{\ell}_{i+\frac12}, \varepsilon_{wen})$}
\\
\cline{2-5}
%${\ell}$
&  $e_{\ell}$ & $\log_2(e_{{\ell}-1}/e_{{\ell}}) $
 &$e_{\ell}$ & $\log_2(e_{{\ell}-1}/e_{{\ell}}) $\\
\hline
 6& 3.5335e+01&-1.9294e+00& 1.4095e-08& 3.9822e+00   \\
   7& 1.3458e+02&-1.9640e+00 &8.9187e-10& 3.9915e+00   \\
   8& 5.2508e+02&-1.9819e+00 &5.6070e-11& 3.9959e+00   \\
   9& 2.0741e+03&-1.9909e+00 &3.5144e-12& 3.9980e+00   \\
   10& 8.2441e+03&-1.9954e+00 &2.1996e-13& 3.9990e+00   \\
   11& 3.2872e+04&-1.9977e+00&  1.3757e-14& 3.9995e+00   \\
   12& 1.3128e+05&-1.9989e+00&  8.6012e-16& 3.9997e+00   \\
   13& 5.2470e+05&-1.9994e+00&  5.3767e-17& 3.9999e+00   \\
\hline
\end{tabular}
$}
 \end{center}
 \caption{Smooth region: next to a jump. Errors $e_{\ell}$ and estimates of the approximation order
   for    $\hat{u}^{{\ell}+1}_{2i+1}=R_4^{WENO}(x^{\ell}_{i+\frac12},\varepsilon)$
   with %parameter estimation
$\varepsilon^2_{lin}$  and
 $\varepsilon^2_{wen}.$ }
%\label{tab:r3a_dis}
  \end{table}

\end{frame}

\begin{frame}
\frametitle{Smooth regions: next to a jump discontinuity $ G^{WENO}_4$}
\begin{table}[h!]
\begin{center}
%\scalebox{0.8}
{$
\begin{tabular}{|c|c|c|c|c|}
\hline
& \multicolumn{4}{|c|}{ \ $\Omega_{d}=[0,0.5]$} \\
%\hline
\cline{2-5}
${\ell}$ & \multicolumn{2}{|c|}{$\uunmezzo -G^{WENO}_4(x^{\ell}_{i+\frac12},
\varepsilon_{lin})$}
& \multicolumn{2}{|c|}{$\uunmezzo -G^{WENO}_4(x^{\ell}_{i+\frac12},
\varepsilon_{wen})$}
\\%[.5cm]
%& \multicolumn{2}{|c|}{$\Rmqunob_2(x^{\ell}_{i+\frac12})$ \hbox{with} $\varepsilon^2_{wen}$}
\cline{2-5}
%${\ell}$
&  $e_{\ell}$ & $\log_2(e_{{\ell}-1}/e_{{\ell}}) $
&$e_{\ell}$ & $\log_2(e_{{\ell}-1}/e_{{\ell}}) $\\
\hline
   6& 2.5632e-02& 6.3092e-02&  2.2037e-07& 2.9429e+00   \\
   7& 2.4536e-02& 3.2641e-02&  2.8658e-08& 2.9717e+00   \\
   8& 2.3987e-02& 1.6608e-02&  3.6532e-09& 2.9859e+00   \\
   9& 2.3712e-02& 8.3781e-03&  4.6114e-10& 2.9930e+00   \\
   10& 2.3575e-02& 4.2078e-03&  5.7925e-11& 2.9965e+00   \\
   11& 2.3506e-02& 2.1086e-03& 7.2583e-12& 2.9982e+00   \\
   12& 2.3472e-02& 1.0555e-03&  9.0839e-13& 2.9991e+00   \\
   13& 2.3455e-02& 5.2803e-04&  1.1362e-13& 2.9996e+00   \\
\hline
\end{tabular}
$}
 \end{center}
 \caption{Smooth region: next to a jump. Errors $e_{\ell}$ and estimates of the approximation order for $\hat{u}^{{\ell}+1}_{2i+1}=G^{WENO}_4(x^{\ell}_{i+\frac12},\varepsilon)$ with parameter estimation
$\varepsilon^4_{lin}$  and
 $\varepsilon^4_{wen}.$
}
  \end{table}
\end{frame}
\begin{frame}
\frametitle{Graphical results: initial resolution $\ell=4,$ final ${\ell}=8$}

\begin{figure}[h!]
  \begin{center}
\begin{tabular}{ccc}
$\varepsilon_{lin}$ &\includegraphics[width=4.5cm]{fig43_r3a_l.eps}\quad   &
\includegraphics[width=4.5cm]{fig43_r3b_l.eps} \\[.2cm]
$\varepsilon_{wen}$ &\includegraphics[width=4.5cm]{fig43_r3a_w.eps}\quad   &
\includegraphics[width=4.5cm]{fig43_r3b_w.eps} \\
%\includegraphics[width=5.8cm]{fig43_r2_wen_lim.eps}  &
%\includegraphics[width=5.8cm]{fig43_r3_wen_lim.eps} \\
& $R^{WENO}_4$ & $G^{WENO}_4$\\
\end{tabular}
  \end{center}
%\caption{Central area of reconstruction (dotted lines) from 17 initial
%  values of the  test function  obtained from the
%  recursive use of  $R^{WENO}_4$ and $G^{WENO}_4$
%  considering  different approximations to
% $\varepsilon_{\text{opt}}$.
%First row:  $\varepsilon_{lin}$. Second row:   $\varepsilon_{wen}.$
%}
\end{figure}
\end{frame}
\begin{frame}
\vskip 2truecm
\centerline{{\Large\coll{\bf....THANK YOU!......}}}
\end{frame}
%
%\begin{frame}
%
%\vs
%\begin{figure}
%  \centering
%  \includegraphics[width=20mm, angle=-90]{rita.eps}
%  \end{figure}
%
%
%
%This research has been accomplished within RITA (Rete ITaliana di Approssimazione)
%
%\centerline{{\large\col{\bf....THANK YOU!......}}}
%\end{frame}
\begin{thebibliography}{10}
\bibitem[Ar\`andiga et a. 04]{AB04}
{ F. Ar\`andiga and A. M. Belda}, Weighted ENO interpolation and applications, Communications in Nonlinear Science
and Numerical Simulation 9 (2004) 187--195.

\bibitem{Ara} F. Ar\`andiga, A. Cohen, R. Donat and N. Dyn, Interpolation and approximation of piecewise smooth functions, SIAM J. Numer. Anal. 43 (2005) 41--57.
\bibitem[Ar\`andiga et a. 10]{ABM10}
{ F. Ar\`andiga, A. M. Belda and P. Mulet},
Point-value {WENO} multiresolution applications to stable image compression, J. Sci. Comput. 43(2) (2010) 158--182.

\bibitem{BR13}
M. Bozzini and M. Rossini, The detection and recovery of discontinuity curves from scattered data, J. Comput. Appl. Math. 240 (2013) 148--162.

\bibitem{BR14} M. Bozzini, L. Lenarduzzi and M. Rossini, Non-regular surface approximation. In:  Floater M., Lyche T., Mazure ML., M{\o}rken K., Schumaker L.L. (eds) Mathematical methods for curves and surfaces. Lecture Notes in Computer Science, vol. 8177, 68--87. Springer, Heidelberg, (2014).

\bibitem{CM05}
A. Crampton and J.C. Mason, Detecting and approximating fault lines from randomly scattered data, Numer. Algorithms 39 (1–3) (2005) 115--130.

\bibitem[De Marchi et al. 18]{DeMP}
S. De Marchi, F. Marchetti and E. Perracchione, Jumping with variably scaled discontinuos kernels (VSDK),
https://www.math.unipd.it/$\tilde{}$demarchi/papers/Draft$\_$DMP$\_$V8.2.pdf

%\bibitem{DD}
%G. Deslauriers and S. Dubuc, Symmetric iterative interpolation, Constr. Approx. 5 (1989) 49--68.

%\bibitem{Eckhoff}
%K.S. Eckhoff, Accurate and efficient reconstruction of discontinuous functions from truncated series expansions, Mathematics of Computation 61(204) (1993), 745--763.

\bibitem[Erb et al. 19]{erb} W. Erb, S. De Marchi, F. Marchetti, E. Perracchione and M. Rossini Shape-Driven Interpolation with Discontinuous Kernels: Error Analysis, Edge Extraction and Applications in MPI, (2019) arXiv:1903.03095 [math.NA].


\bibitem{Gout08}
C. Gout, C. Le Guyader, L. Romani, A.-G. Saint-Guirons, Approximation of surfaces with fault(s) and/or rapidly varying data, using a segmentation process, Dm-splines and the finite element method, Numer. Algorithms 48 (1–3) (2008) 67--92.

\bibitem[Guo et al. 17a]{GJ17a}
J. Guo and J.-H. Jung, A RBF-WENO finite volume method for hyperbolic conservation laws with the monotone polynomial interpolation method,
Appl. Numer. Math., 112 (2017) 27--50.

\bibitem[Guo et al. 17b]{GJ17b}
J. Guo, J.-H. Jung, Radial basis function ENO and WENO finite difference methods based on the optimization of shape parameters,
J. Sci. Comput., 70 (2017) 551--575.


\bibitem{HEOC2}
A. Harten, B. Engquist, S. Osher, S. Chakravarthy, Some results on uniformly high-order accurate essentially nonoscillatory schemes,
Applied Numerical Mathematics 2 (1986) 347--377.


\bibitem{Harten87}
A. Harten, B. Engquist, S. Osher, S. Chakravarthy, Uniformly high order accurate essentially non-oscillatory schemes III,
J. Comput. Phys. 71 (1987) 231--303.

\bibitem{Harten89} A. Harten, ENO schemes with subcell resolution, J. Comput. Phys. 83 (1989) 148--184.


\bibitem{JS96}
G.-S. Jiang and C-W Shu, Efficient implementation of weighted {ENO} schemes, J. Comput. Phys. 126 (1996) 202--228.


\bibitem{Jung}
J.-H. Jung, A note on the Gibbs phenomenon with multiquadric radial basis functions, Appl. Numer. Math. 57 (2007), 213--219.


\bibitem{keller}
{
E. Isaacson and H. B. Keller, Analysis of Numerical Methods. Wiley, New York, 1966.
}


%\bibitem{yoon}
%B.-G. Lee, Y. J. Lee and J. Yoon, Stationary binary subdivision schemes using
%radial basis function interpolation, Adv. Comput. Math. 25 (2006) 57--72.

\bibitem{LS17}
L. Lenarduzzi and R. Schaback, Kernel-based adaptive approximation of functions with discontinuities, Appl. Math. Comput. 307 (2017) 113--123.

\bibitem{LiuChan94}
X.-D. Liu, S. Osher and T. Chan, Weighted essentially non-oscillatory schemes, J. Comput. Phys. 115 (1994) 200--212.

\bibitem{romani}[Romani et al. 18] L. Romani, M. Rossini, D. Schenone, Edge detection methods based on RBF interpolation, J. Comput. Appl.
Math. 349 (2019), 532--547.

\bibitem{shu} C. W Shu, Essentially non-oscillatory and weighted essentially non-oscillatory schemes for hyperbolic
conservation laws. In: Quarteroni, A. (ed.) Advanced Numerical Approximation of Nonlinear Hyperbolic
Equations. Lecture Notes in Mathematics, vol. 1697,  325--432. Springer, Berlin, Heidelberg (1998)

\end{thebibliography}

\end{document}


