
Assume  that the domain segmentation phase was successful, so that  we have got a collection~$\Cal I_j$ of triangles that correctly define the various subdomains $\Omega_j$ of~$\Omega$, where the sampled function~$f$ is continuous. The next phase consists in actually recovering~$f$.   In order to produce an accurate result, the recovery process should be aware of the discontinuities of~$f$.


To reconstruct~$f$ from the data~$(X, F)$ means to find an interpolant~$s$ that can predict with good accuracy the values of~$f$ at any given set~$E\subset\Omega$ of {\em evaluation points}.   If there is some way to separate any set $E$ into a collection of sets~$\{E_j\}_{j=1}^M$ such that $E_j\subset\Omega_j$ for each~$j$, then the reconstruction can easily be performed, by considering each subdomain separately.

Let $X_j\subset X$ be the set of vertices of the triangles~$\Cal I_j$, i.e., the set of all the data sites that belong to~$\Omega_j$, and $F_j$ its corresponding set of data values.  Assume to have produced for each $j=1,2,\dots, M$ a regular interpolant $s_j$ of the data~$(X_j, F_j)$, either by direct interpolation with some appropriate kernel function~$\Phi$, or by using other techniques like the {\em partition of unity} method~(Wendland~\cite[wendland_2002]).  Then the reconstructed version~$s$ of the function~$f$ can be piecewise defined as
$$
s(e)\coloneq\cases{s_{1\!\.}(e) & if $e\in E_1$ \cr
             s_{2\!\.}(e) & if $e\in E_2$ \cr
             \quad\hbyw{3ex}{0em}\smash{\vdots}&\quad\smash{\vdots}  \cr
             s_{M\!\.}(e) & if $e\in E_M$.}
$$


We therefore only need some way to assign to each evaluation point~$e\in E$ its domain~$\{\Omega_j\}_{j=1}^N$.  The only information available is that for each~$j$ the data sites~$X_j$ belong to~$\Omega_j$.  This information must be used to predict where to locate each  point of the given set~$E$.  Of course we can't expect to perfectly separate~$E$ into subsets~$E_j$ such that $E_j$ is  fully contained inside~$\Omega_j$ for each~$j$, with the available knowledge.


For the classification of the evaluation points we use an algorithm from the field of machine learning, called {\em support vector machine} (SVM), which belongs to the class of the supervised learning algorithms (Christmann and Steinwart~\cite[christmann-steinwart_2008]).  The basic version of the algorithm works with only two classes. So, in order to describe it, we now suppose that our set~$X$ of data sites has been divided into two sets only, namely~$X_1, X_2$.  The  general case of an arbitrary number~$M$ of classes will be discussed later. 

A SVM works in the following way.  First the points are sent through a map~$\Theta$, called $\.${\em feature map}\fnote{In machine learning a {\em feature} is an individual measurable property or characteristic of a phenomenon being observed.  Here the features are the coordinates of the points, that get transformed by~$\Theta$ into a different set of coordinates.}$\!\!$,$\.$ into some higher dimensional vector space~$V\!$, and then a hyperplane that best separates $\Theta(X_1)$ from~$\Theta(X_2)$ is found there.  The hyperplane is described implicitly as the zero set of the function
$$
h(x) \coloneq \langle w, \Theta(x)\rangle_V + b,\quad x\in\Omega, \eqmark[hyperplane]
$$
where $w$ and~$b$ are two vectors of~$V\!$.\,  The orthogonal vector~$w$ and the offset vector~$b$ are determined such that the geometric margin from the hyperplane to both $\Theta(X_1)$ and~$\Theta(X_2)$ is maximised, either in a strict way if $\Theta(X_1)$ and~$\Theta(X_2)$ are linearly separable, or in a weaker form.
 In any case, the problem of determining $w$ and~$b$ can be converted into an optimisation problem whose formulation has the peculiarity of depending on the original data~$X$ only through the scalar products between pairs of the associated transformed points~$\Theta(X)\subset V\!$.\,

Specifically, one has to find the coefficients~$\range \lambda_N/$ that maximise the quantity
$$
\displaylines to \hsize{
\sum_{i=1}^N\lambda_i - {1\over2}\sum_{i=1}^N\sum_{j=1}^N\lambda_i z_i\,\langle \Theta(x_i), \Theta(x_j)\rangle_V\, \lambda_j z_j,\quad\hbox{where } z_k\coloneq \cases{\phantom-1 & if $x_k \in X_1$\cr -1 & if $x_k\in X_2$},\cr
\hbox{subject to the constraints}\hfill\eqmark[optimisation]
\cr
\left\{\eqalign{&\textstyle\sum_{i=1}^N\lambda_i z_i = 0\cr
                & 0\leq\lambda_i\leq C\quad\mathbox{for all~$i$}.}\right.
}
$$
Here the box constraint~$C$ is a cost factor which indicates how much weight should be assigned to misclassified points.  A value~$C=\infty$, for instance, tells the algorithm to attempt to find a strict separation of the data, if possible.
After having obtained the coefficients~$\{\lambda_i\}_{i=1}^N$, the solution vectors $w$ and~$v$ are computed as
$$
w = \sum_{j=1}^N \lambda_j z_j x_j, \qquad b =\sum_{j=1}^N\lambda_j z_j\langle \Theta(x_i), \Theta(x_j)\rangle_V\, - \.z_i, 
$$
where $i$ is an index such that~$0<\lambda_i<C$.
The above optimisation problem can be efficiently solved by some quadratic programming algorithm provided that the scalar products~$\langle \Theta(x_i), \Theta(x_j)\rangle_V$ can be efficiently computed.


Here the {\em kernel trick} comes into play.  By virtue of Mercer theorem any symmetric positive definite kernel~$\Phi:\Omega\times\Omega\to\R$ can be decomposed as
$$
\Phi(x, y) = \langle \Theta(x), \Theta(y)\rangle_V\. = \sum_{i=1}^\infty\Theta_i(x)\Theta_i(y),\quad x,y\in\Omega, 
$$
where each~$\Theta_i$ is an eigenfunction of the integral operator~$f\mapsto\int_\Omega f(x)\Phi(x,y)\,dx$.  Therefore, if the feature map~$\Theta$ is taken from the decomposition of some kernel~$\Phi$, then the scalar products of pairs of points in the higher dimensional (possibly infinite dimensional) vector space~$V$ can be efficiently computed,  regardless of the dimension of~$V\!$,\, by simply evaluating the kernel function on the corresponding pairs of points of~$X$.  Notice moreover that the whole computations can be performed without even knowing who the feature map~$\Theta$ is, but simply by evaluating some known kernel~$\Phi$.

One example of kernel used in the context of SVMs is the {\em polynomial} kernel of degree~$d$, whose expression is
$$
\Phi(x,y)= (\langle x, y\rangle + c)^d,
$$
where $c\geq0$ is a free parameter trading off the influence of higher-order terms versus lower-order terms in the polynomial. In the case $c=0$ and~$d=1$ this kernel  is just the scalar product, and the corresponding feature map~$\Theta$ is the identity map---using this kernel is equivalent to performing a linear separation of the data in the original domain~$\Omega$.

Anyhow, the most used kernel for SVM classification is the {\em Gaussian} kernel
$$
\Phi(x,y) = \phi(ε\norm{x-y}) = e^{-ε^2\norm{x-y}^2},
$$
previously defined in Example~\ref[gaussianex].  It is known that, differently from the polynomial kernel, the integral operator associated to the Gaussian kernel has an infinite set of eigenvalues (Ha Quang, Niyogi and Yao~\cite[ha_quang-niyogi-yao_1970]), therefore the feature map~$\Theta$ into which $\Phi$ decomposes maps the data~$X$ into a vector space~$V\!$ of infinite dimension.


At the end of the training process, the SVM has learnt a possibly nonlinear decision boundary in the space~$\Omega$ by learning a linear decision boundary in the space~$V\!$.\,
The decision function~$h$ defined in~\ref[hyperplane] can now be used to classify the evaluation points~$E$ into the two classes $E_1$ and~$E_2$, by defining these two classes in the following way:
$$
E_1\coloneq \{\,e\in E\,:\>h(e)\geq 0\,\},\qquad
E_2\coloneq \{\,e\in E\,:\>h(e)<0\,\} = E\setminus E_1.
$$

In the general case, when we have an arbitrary number~$M$ of sets~$\range X_M/$, a multiclass SVM can be implemented by using $M-1$ binary SVMs.  In fact, if $X_1, X_2,\dots X_M$ are the sets of data sites,  we can first train a SVM with the two classes~$X_1$ and~$X_2\cup\dots\cup X_M$ to produce a first decision function~$h_1$; then we train another SVM with the two classes $X_2$ and~$X_3\cup\dots\cup X_M$ to produce a second decision function~$h_2$; and so on, until we have produced the function~$h_{M-1}$.  After that, any given set of evaluation points~$E$ can be split into the $M$ sets $\range E_M/$ by
$$
\eqalign{
&E_1 \coloneq \{\,e\in E\,:\>\> h_1(e) \geq 0\,\}, \cr
&E_2 \coloneq \{\,e\in E\setminus E_1\,:\>\> h_2(e) \geq 0\,\},\cr
&E_3 \coloneq \{\,e\in E\setminus(E_1\cup E_2)\,:\>\> h_3(e) \geq 0\,\},\cr
&\,\,\vdots  \qquad\qquad\qquad\vdots\cr
&E_{M-1} \coloneq \{\,e\in E\setminus(E_1\cup E_2\cup\dots\cup E_{M-2})\,:\>\> h_{M-1}(e) \geq 0\,\}\cr
&E_M \coloneq E\setminus(E_1\cup E_2\cup\dots\cup E_{M-1}).
}
$$


Actually, not all the points of each set~$X_j$ seem to be  necessary to train the SVM.  The points that really count are those located at the boundary of each set~$X_j$.  Since we have at our disposal the triangulation~$\Cal I_j$ of each set~$X_j$, the boundary points can be easily located by taking the boundary of those triangulations.  If the training sets~$\smash{\{X_j\}_{j=1}^M}$ get replaced by the corresponding sets of boundary points, then the optimisation problem~\ref[optimisation] should require less computations to reach the solution, and the training process may be faster.



\nonum\secc Examples


