\newtoks\tmptoks
\def\Cal#1{{\cal #1}}
\def\Bm#1{\hbox{$\boldmath #1$}}
\leftskip 1cm
\rightskip 1cm

\fontfam[lm]\sans


\slides
\typosize[18/33]

\gdef\_titfont{\typosize[25/38]\bf \Blue}
\gdef\_subtitfont{\typosize[20/35]}



\tit  ON THE PROBLEM OF RECOVERING\nl DISCONTINUOUS FUNCTIONS\nl FROM SCATTERED DATA
\subtit  Matteo Caoduro\nl {\it Relatore:\,} Prof.ssa Milvia Francesca Rossini\nl 18 marzo 2021

\pg;

\null\vskip 1.5cm
* Interpolazione basata su kernel
\medskip
* Classificazione dei dati
\medskip
* Individuazione delle discontinuità e ricostruzione

\pg;

* $X =\{x_1,x_2,\dots,x_N\}$ insieme di locazioni
* $F = \{f_1,f_2,\dots,f_N\}$ insieme di valori campionati da una funzione~$f$
* $\Cal S$ spazio lineare di funzioni~$s:Ω\to ℝ$
\smallskip
* {\bf Problema:} Trovare $s\in\Cal S$ tale che $s(x_j) = f_j$ per ogni~$j$
* $s = \sum_{j=1}^N \alpha_j B_j$, se $\Cal B = \{B_1,B_2,\dots B_N\}$ è una base di $\Cal S$

 
* Sistema lineare $\Bm A \Bm\alpha = \Bm f$,
$$ \eqalign{
&\Bm A_{i, j} = B_j(x_i) \cr
\Bm \alpha &= (\alpha_1,\alpha_2,\dots,\alpha_N)^T \cr
\Bm f &= (f_1,f_2,\dots, f_N)^T
} $$



\pg.
