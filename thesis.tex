\useOpTeX  % We are using OpTeX, not LaTeX :)

\load[thesis-mac]

%\fnotelinks \Black \Black
\hyperlinks \Black \Black
\outlines 1

\enlang
\fontfam[TypewriterMT]
\fontfam[BaskervilleMT]
\typosize[12.2/14.42]
\load[patches, garamond-math-fix, micro]
%\OpTeX
% We must ensure that . , ; etc are from the same font in text and math mode.

\font\symbols=Diamond % MonotypeSorts
\def\QED{\hbox{\symbols\resizethefont \,❖}}
\font\boldmathfont=EBGaramond-SemiBoldItalic
\famvardef\tt{\TypewriterMT\setff{-liga; -tlig; embolden=1.1}\typoscale[870/]\rm}


\margins/1 a4 (103,95,125,150)pt


\parindent 1em
\iindent=\parindent \ttindent=\parindent
\parskip 1.5ex
\rulewidth 0.13ex

\picdir={pic/}

\bgroup
\let\_mtext\ignoreit
\nonum\notoc\chap Contents
\egroup
\tocpage


\bgroup
\let\_mtext\ignoreit
\nonum\chap Introduction
\egroup

%%%%%%%%%%%%%%%%%%%%%%%%%%%%%%%%%%
\chap Kernel-based interpolation

\sec The interpolation problem
\input the-interpolation-problem

\sec[positivesec] Positive definite functions
\input positive-definite-functions

\sec Native spaces
\input native-spaces
%  Sobolev spaces

\sec[errorsec] Error estimates
\input error-estimates

\sec Optimality
\input optimality

\sec[addingsec] Adding data
\input adding-data

%%%%%%%%%%%%%%%%%%%%%%%%%%%%%%%%%%%%%%%%%%%%%%
\chap Dealing with discontinuities

\sec[buildingsec] Building a regularity classifier
\input building-a-regularity-classifier

\sec Properties of the classifier
\input properties-of-the-classifier

\sec Edge detection

\sec Reconstruction

\sec Some examples



%\sec Gibbs and Runge phenomena
%\sec Variably scaled kernels
%\sec Fake nodes
%\sec Partition of unity



% manual tuning of parameter


% Separate good behaving data set---the ones that comes from a regular function---from bad behaving ones---the ones that come from a discontinuous function.

% Conclusions:  Another approach worth investigating is to use  machine learning on the data set to build the classification function.  Maybe a black box machine learning techniques can perform better than the tool appositely developed.

% The updated version of the thesis can be retrieved at...


% BIBLIOGRAPHY
\bgroup
\let\_mtext\ignoreit
\nonum\chap Bibliography

\def\_opwarning#1{}
\usebib/s (iso690) biblist
\egroup

\bye
