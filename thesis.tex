\useOpTeX  % We are using OpTeX, not LaTeX :)

\load[thesis-mac]

%\fnotelinks \Black \Black
\hyperlinks \Black \Black
\outlines 1

\enlang
\fontfam[BaskervilleMT]
\fontfam[EBGaramond]
\fontfam[TypewriterMT]
\BaskervilleMT
\typosize[12.2/14.42]
\load[patches, garamond-math-fix, micro]
%\OpTeX
% We must ensure that . , ; etc are from the same font in text and math mode.
\fontdef\rmfixex{\rm}
\fontdef\itfixed{\it}
\fontdef\bffixed{\bf}
\fontdef\bifixed{\bi}

\font\symbols=Diamond % MonotypeSorts
\def\QED{\hbox{\symbols\resizethefont \,❖}}
\famvardef\bm{\EBGaramond\setff{-liga; -kern}\semibold\bi}
\famvardef\bmu{\EBGaramond\setff{-liga; -kern}\semibold\bf}
\famvardef\tt{\TypewriterMT\setff{-liga; -tlig; embolden=1.1}\typoscale[870/]\rm}


\margins/1 a4 (103,95,145,130)pt


\parindent 1em
\iindent=\parindent \ttindent=\parindent
\parskip 1.5ex
\rulewidth 0.13ex

\picdir={pic/}


\let\omtext\_mtext
\def\_mtext#1{CONTENTS}
\let\uppercase\ignoreit
\nonum\notoc\chap Contents
\let\_mtext\omtext
\let\uppercase\_uppercase
\bigskip
\maketoc
\pageno=0
\footline={}
\vfill\eject


\def\_mtext#1{INTRODUCTION}
\let\uppercase\ignoreit
\nonum\chap Introduction
\let\_mtext\omtext
\let\uppercase\_uppercase


%\comment
%%%%%%%%%%%%%%%%%%%%%%%%%%%%%%%%%%
\chap Kernel-based interpolation

\def\sectitle{The interpolation problem}
\ea\sec\sectitle

\input the-interpolation-problem

\sec[positivesec] Positive definite functions
\input positive-definite-functions

\sec Native spaces
\input native-spaces
%  Sobolev spaces

\sec[errorsec] Error estimates
\input error-estimates

\sec Optimality
\input optimality

\sec[addingsec] Adding data
\input adding-data

%\endcomment
%%%%%%%%%%%%%%%%%%%%%%%%%%%%%%%%%%%%%%%%%%%%%%
%\chap Dealing with discontinuities
\chap Data classification

\sec[buildingsec] Building a regularity classifier
\input building-a-regularity-classifier

\sec Invariance of the regularity vector
\input invariance-of-the-regularity-vector

\sec Numerical examples
\input numerical-examples

\chap Dealing with discontinuities
% two-dimensional discontinuities

\sec Edge detection ???

\sec Reconstruction

\sec Some examples



%\sec Gibbs and Runge phenomena
%\sec Variably scaled kernels
%\sec Fake nodes
%\sec Partition of unity



% manual tuning of parameter


% Separate good behaving data set---the ones that comes from a regular function---from bad behaving ones---the ones that come from a discontinuous function.

% Conclusions:  Another approach worth investigating is to use  machine learning on the data set to build the classification function.  Maybe a black box machine learning techniques can perform better than the tool appositely developed.

% The updated version of the thesis can be retrieved at...


% BIBLIOGRAPHY
%\let\_mtext\ignoreit
\def\_mtext#1{BIBLIOGRAPHY}
\let\uppercase\ignoreit
\nonum\chap Bibliography
\let\_mtext\omtext
\let\uppercase\_uppercase
\def\_opwarning#1{}
\usebib/s (simple) biblist


\bye
